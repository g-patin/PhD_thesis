% !TEX root = main.tex

\newgeometry{top=2.170cm,
            bottom=3.510cm,
            inner=2.1835cm,
            outer=2.1835cm,
            ignoremp}


\begin{titlepage}
   \begin{center}
       \vspace*{4cm}

       \textbf{\Huge Improved microfading device for risk assessment of colour change in Van Gogh's works}

           
       \vspace{7cm}

       \textbf{\Large Gauthier Patin}          
  
            
   \end{center}
\end{titlepage}




% Copyright page




\clearpage
\thispagestyle{empty}
\null%
\label{thesis:colophon}
\vfill
\pdfbookmark[1]{Colophon}{thesis:colophon}
Written in 2018--2023 by Gauthier Patin 

ISBN: 978909037535

Copyright \copyright \hspace{0.1cm}2023 by Gauthier Patin

Printed by Proefschriftmaken, \url{https://www.proefschriftmaken.nl}\\


\textbf{Caveat} \\
I must note that throughout this thesis I have used several excerpts of others' text, images and code, which I have always been careful to mark as such.
While I am allowed to use these excerpts under legal \emph{fair use} doctrines in many countries and more specifically by the citation right (\emph{citaatrecht}) in \href{http://wetten.overheid.nl/jci1.3:c:BWBR0001886&hoofdstuk=I&paragraaf=6&artikel=15a&z=2015-07-01&g=2015-07-01}{Article 15a of the Dutch Copyright Law} (\emph{Auteurswet}), it does not follow that you are also free to use these excerpts for any purpose.\\

\textbf{Colophon} \\
This thesis was typeset with the help of \LaTeX{} and to a large extent uses the source code provided by Ken Arroyo Ohori for his PhD thesis (\url{https://github.com/kenohori/thesis}). Most of the figures were created using Matplotlib package inside Jupyter notebooks, or Inkscape.

The source code of this thesis is available at: \url{https://github.com/g-patin/PhD_thesis}

A digital version of this dissertation is available on the Digital Academic Repository of the University of Amsterdam (\url{https://dare.uva.nl}) as well on my PhD website (\url{https://microfadingphd.wordpress.com}). 

The data that have been used to create the figures in Chapters 3, 4, and 5 have been stored on a Zenodo repository (). \\

\textbf{Cover} \\
The front and back covers are based on the results of light ageing experiments conducted during this PhD using the methodology described in Chapter 4 (section \ref{sec:DL_methodology}) of this dissertation. Both covers display the results of daylight exposure on an eosin paint-out (PO098). The front cover shows visualization of colour patches as the light dose increases from 0 to 120 MJ/m\textsuperscript{2}, while the back cover shows differences in the reflectance spectra as the light dose increases from 0 to 120 MJ/m\textsuperscript{2}. A value of 120 MJ/m\textsuperscript{2} roughly corresponds to 50 years of continuous exposure in the galleries of the Van Gogh Museum (10 hours per day at 50 lux).




% Official title
\begin{titlepage}
\begin{center}
    \vspace*{3cm}

    \textbf{\Huge Improved microfading device}\\
    \vspace{1cm}
    \textbf{\Huge for risk assessment of colour change}\\
    \vspace{1cm}
    \textbf{\Huge in Van Gogh's works}

           
    \vspace{5cm}
    

%% Skip space as in half-title
\vspace*{4\baselineskip}



{\LARGE ACADEMISCH PROEFSCHRIFT}

\medskip
\large
{ter verkrijging van de graad van doctor aan de Universiteit van Amsterdam \\
op gezag van de Rector Magnificus \\
prof. dr. ir. P.P.C.C. Verbeek \\
ten overstaan van een door het College voor promoties ingestelde commissie, \\
in het openbaar te verdedigen op vrijdag XX XXXX 2024, te XX:00 uur \\
}

\medskip

door

\medskip

%% Print the full name of the author.
\makeatletter
{\Large Gauthier {\scshape Patin}}
\makeatother

\medskip

geboren te Grande-Synthe, Frankrijk.

\end{center}
\end{titlepage}


% Official verso
\clearpage
\thispagestyle{empty}
\null%
\label{thesis:committee}
\vfill
\pdfbookmark[1]{Doctoral committee}{thesis:committee}

\LARGE

\noindent This dissertation has been approved by the

\medskip\noindent
\begin{tabular}{@{}ll@{}}
  \quad{}promotor: & prof.\ dr.\ E.\ Hendriks \\
  \quad{}copromotor: & prof.\ dr.\ K.J.\ van den Berg \\
  \quad{}copromotor: & prof.\ dr.\ R.G.\ Erdmann \\
\\
\multicolumn{2}{@{}l@{}}{Composition of the doctoral committee:} \\
\\
  \quad{}prof.\ dr.\ M.R.\ van Bommel & Universiteit van Amsterdam \\
  \quad{}prof.\ dr.\ S.\ Woutersen & Universiteit van Amsterdam \\
  \quad{}prof.\ dr.\ K.\ Keune & Universiteit van Amsterdam  \\  
  \quad{}prof.\ dr.\ S.\ Woutersen & Universiteit van Amsterdam \\
  \quad{}prof.\ dr.\ J.\ Dik & TU Delft \\
  \quad{}dr.\ J.\ del Hoyo-Meléndez & National Museum Krakow \\
  \quad{}dr.\ A.\ Bülow & Stedelijk Museum Amsterdam\\
\\

\end{tabular}

This research was supported by the AXA Research Fund and the Van Gogh Museum-ASML Partnership in Science, the Rijksdienst voor het Cultureel Erfgoed, and the Rijksmuseum Amsterdam. 



