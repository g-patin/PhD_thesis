% !TEX root = main.tex

\newgeometry{top=2.170cm,
            bottom=3.510cm,
            inner=2.1835cm,
            outer=2.1835cm,
            ignoremp}

\addchap[Abstract - Samenvattig - Résumé]{Abstract}

\textbf{\textcolor{blue}{Improved microfading device for risk assessment of colour change in Van Gogh's works}}\\

Over the past three decades, conservators and curators at the Van Gogh Museum have gathered increasing evidence of colour changes having occurred in Van Gogh’s paintings. Numerous research projects on colour change in Van Gogh’s works, conducted since the 1990s, have greatly deepened our knowledge on the issue and are reviewed in Chapter \ref{ch:ch1_general-intro}. For instance, the light-sensitive pigments responsible for these colour changes have been identified and their degradation mechanisms investigated. Additionally, light-ageing experiments on paint-outs were conducted and several attempts to estimate the original appearance of paintings carried out. Yet, our actual knowledge of the timescale of this problem is rather limited. To what extent has the observed fading and discolouration in Van Gogh’s paintings now stabilised? And if still ongoing, how fast does it occur? In other words, can we expect to see colour changes taking place in Van Gogh’s paintings over the next ten years or so? The work performed during the timeframe of this PhD contributes towards providing answers to these questions, by supplying reliable data for light fading risk assessment on which to base decisions regarding the long-term preservation of the paintings. \\

In the late 1990s, researchers in the \gls{USA} and \gls{UK} developed devices to monitor light-induced colour change at a microscale level. This technique, known as microfading test or microfadeometry and thoroughly described in Chapter \ref{ch:ch2_MFT}, enables lightfastness analyses on objects, helping museum professionals to make decisions regarding the illumination of their collections. Nevertheless, the technique presents a number of limitations, which hinders reliable interpretation of the data. The specific aim of this PhD project is to improve microfading analyses on technical and methodological levels so that it can take us closer towards reliable colour change prediction. \\ 

In this respect, the PhD project focused on three topics related to microfadeometry. The first one aimed to make technical progress regarding the way microfading analyses are being performed (Chapter \ref{ch:ch3_MFT-advances}, section \ref{sec:stereo-MFT}). The second was intended to improve our knowledge on the rise of temperature at the surface of materials irradiated by a microfading beam (Chapter \ref{ch:ch3_MFT-advances}, section \ref{sec:MFT-Temp}). The third topic aimed to evaluate the accuracy of microfading data by assessing the validity of the reciprocity principle on several appropriate paint samples (Chapter \ref{ch:ch3_MFT-advances}, section \ref{sec:MFT-reciprocity}) and by comparing microfading data with traditional light-ageing methods (Chapter \ref{ch:ch4_light-ageing}). \\

The technical improvements accomplished led to the development of a new system called the stereo-MFT, which is presented in Chapter \ref{ch:ch3_MFT-advances}, section \ref{sec:stereo-MFT}. This new device uses a stereomicroscope as its central element for which the optics – in combination with a high-quality microscope camera – enables a better characterisation of the microfading beam. Ultimately, it improves the accuracy of our estimation of the amount of energy received by the sample, suggesting that dose-response microfading analyses can be performed with precision. Applications of the device on paint and textiles samples furthermore enabled us to outline several of its limitations that could potentially be resolved during future research projects to bring further improvements (Chapter \ref{ch:ch5_applications-MFT}). \\

The use of an infrared temperature sensor successfully provided surface temperature measurements of paint samples irradiated by a microfading beam (Chapter \ref{ch:ch3_MFT-advances}, section \ref{sec:MFT-Temp}). The results showed an increase of temperature significantly higher than most values reported in the literature. An increase of temperature ranging from 4\unit{\degreeCelsius} to 13\unit{\degreeCelsius} has been observed for the colours most frequently analysed with microfadeometry: yellow, oranges and reds. Additionally, the implementation of a mathematical model to simulate temperature changes and heat transfer inside irradiated materials proved to be a valuable tool to understand the influence of certain parameters on the heat transfer behaviour inside paint layers. For instance, parameters related to the light beam (power, size) or to the materials (thermal conductivity, specific heat capacity, heat transfer coefficient, density, etc.) can be studied. \\ 

The reliability of microfading data is directly related to the reciprocity principle. Research on this principle necessitates accurate characterization of the microfading beam in terms of spectral power distribution and light dose exposure, either in \unit{\lux\hour} or in \unit{\mega\joule\per\square\metre}. The reciprocity principle experiments (Chapter \ref{ch:ch3_MFT-advances}, section \ref{sec:MFT-reciprocity}) and the comparison with other light-ageing methods (Chapter \ref{ch:ch4_light-ageing}) confirmed failures of the principle and a lack of correlation with colour change occurring under normal conditions for half of the samples. Interpretation of the data enabled us to propose some technical and methodological ways to bypass this problem such as the use of colour change rate as a more appropriate metric for applying microfading data to works of art in the framework of lighting policies. \\

In a global way, all results obtained in this thesis converge on a single conclusion: microfadeometry is an effective and efficient technique for the detection and analysis of colour change phenomena. At the same time, it has not yet revealed its full potential, which, in my opinion, lies in its ability to correlate its point-wise results with other analytical data. Given the increasing importance of imaging techniques in cultural heritage science over the past decades, it seems relevant to orient microfading analyses in this direction with the intention of creating a computer-controlled motion system whereby microfading spots could be assigned $x$ and $y$ coordinates to precisely locate them on objects. While allowing a connection of microfading results with other data, this could also lead to image-based colour change prediction.

\newpage
\textbf{\textcolor{blue}{\LARGE Samenvattig}}\\

\textbf{\textcolor{blue}{Verbeterd microfading apparaat voor risicobeoordeling van kleurverandering in de werken van Van Gogh}}\\


In talrijke onderzoeksprojecten hebben restauratoren en conservatoren van het Van Gogh Museum in de afgelopen drie decennia steeds meer bewijs gevonden voor kleurveranderingen in de schilderijen van Van Gogh. Deze bevindingen hebben onze kennis over dit onderwerp aanzienlijk verdiept. De lichtgevoelige pigmenten die verantwoordelijk zijn voor deze kleurveranderingen zijn bijvoorbeeld geïdentificeerd en hun afbraakmechanismen zijn onderzocht. Daarnaast zijn lichtverouderingsexperimenten op verfmonsters uitgevoerd en verschillende pogingen om het oorspronkelijke uiterlijk van schilderijen te schatten. Toch is onze feitelijke kennis van de tijdschaal van dit probleem vrij beperkt. In hoeverre is de waargenomen verbleking en verkleuring van Van Gogh's schilderijen nu gestabiliseerd? En als het nog steeds doorgaat, hoe snel gaat het dan? Kunnen we bijvoorbeeld verwachten dat de kleurveranderingen op Van Gogh's schilderijen de komende ongeveer tien jaar zullen plaatsvinden? Deze aspecten worden in hoofdstuk \ref{ch:ch1_general-intro} besproken. Het werk dat in het kader van dit proefschrift is uitgevoerd, draagt bij aan de beantwoording van deze vragen door betrouwbare gegevens te leveren voor de beoordeling van het risico op verbleking en andere kleurveranderingen, op basis waarvan beslissingen kunnen worden genomen over de conservering van de schilderijen op de lange termijn.\\


Eind jaren 1990 ontwikkelden onderzoekers in de VS en het VK instrumenten om door licht veroorzaakte kleurverandering op microscopisch niveau te monitoren. Deze techniek, bekend als ‘microfading test’ of ‘microfadeometry’ en uitvoerig beschreven in hoofdstuk \ref{ch:ch2_MFT}, maakt analyses van de lichtechtheid van objecten mogelijk (vanaf hier genoemd microfadeometry of microfading analyses) en helpt ‘museum professionals’ om beslissingen te nemen over een optimale belichting van hun collecties. De techniek heeft echter een aantal beperkingen die een betrouwbare interpretatie van de gegevens in de weg staan. Het specifieke doel van dit PhD-project is om microfading analyses op technisch en methodologisch niveau te verbeteren, zodat het ons dichter bij een betrouwbare voorspelling van kleurveranderingen kan brengen. \\

Hiervoor richtte het PhD-project zich op drie onderwerpen met betrekking tot microfadeometry. De eerste was gericht op het boeken van technische vooruitgang met betrekking tot de manier waarop microfading analyses worden uitgevoerd (Hoofdstuk \ref{ch:ch3_MFT-advances}, sectie \ref{sec:stereo-MFT}). Het tweede was gericht op het verbeteren van onze kennis over de temperatuurstijging aan het oppervlak van materialen die bestraald worden door een microfading lichtbundel (Hoofdstuk \ref{ch:ch3_MFT-advances}, sectie \ref{sec:MFT-Temp}). Het derde onderwerp was bedoeld om de nauwkeurigheid van microfading data te evalueren door de geldigheid van het reciprociteitsprincipe te beoordelen op verschillende geschikte verfmonsters (hoofdstuk \ref{ch:ch3_MFT-advances}, sectie \ref{sec:MFT-reciprocity} en door microfading data te vergelijken met traditionele lichtverouderingsmethoden (hoofdstuk \ref{ch:ch4_light-ageing}).\\


De bereikte technische verbeteringen leidden tot de ontwikkeling van een nieuw systeem genaamd de stereo-MFT, dat wordt gepresenteerd in hoofdstuk \ref{ch:ch3_MFT-advances}, sectie \ref{sec:stereo-MFT}. Dit nieuwe instrument maakt gebruik van een stereomicroscoop als centraal element waarvan de optiek - in combinatie met een hoogwaardige microscoopcamera - een betere karakterisering van de microfading lichtbundel mogelijk maakt. Uiteindelijk verbetert het de nauwkeurigheid van onze schatting van de hoeveelheid energie die het oppervlak van object ontvangt, wat suggereert dat dosis-respons microfading analyses met precisie kunnen worden uitgevoerd. Toepassingen van het apparaat op verf- en textielmonsters vermijden ons bovendien in staat om een aantal beperkingen aan te geven die mogelijk opgelost kunnen worden tijdens toekomstige onderzoeksprojecten (Hoofdstuk \ref{ch:ch5_applications-MFT}).\\

Het gebruik van een infrarood temperatuursensor leverde met succes metingen op van de oppervlaktetemperatuur van verfmonsters die bestraald werden door een microfading lichtbundel (hoofdstuk \ref{ch:ch3_MFT-advances}, sectie \ref{sec:MFT-Temp}). De resultaten toonden een temperatuurstijging die aanzienlijk hoger was dan de meeste waarden die in de literatuur werden gerapporteerd. Een temperatuurstijging van 4 tot 13 graden Celsius werd waargenomen voor de kleuren die het meest werden geanalyseerd met microfadeometrie: geel, oranje en rood. Daarnaast bleek de implementatie van een wiskundig model om temperatuurveranderingen en warmteoverdracht binnen bestraalde materialen te simuleren een waardevol hulpmiddel om de invloed van bepaalde parameters op het warmteoverdrachtsgedrag binnen verflagen te begrijpen. Zo kunnen bijvoorbeeld parameters met betrekking tot de lichtbundel (vermogen, grootte) of de materialen (warmtegeleidingsvermogen, specifieke warmtecapaciteit, warmteoverdrachtscoëfficiënt, dichtheid, enz.) worden bestudeerd. \\

De betrouwbaarheid van gegevens over microfading houdt rechtstreeks verband met het reciprociteitsprincipe. Onderzoek naar dit principe vereist een nauwkeurige karakterisering van de microfadingsbundel in termen van spectrale vermogensverdeling en blootstelling aan lichtdosis, hetzij in \unit{\lux\hour} of in \unit{\mega\joule\per\square\metre}. De experimenten met het reciprociteitsprincipe (hoofdstuk \ref{ch:ch3_MFT-advances}, sectie \ref{sec:MFT-reciprocity}) en de vergelijking met andere lichtverouderingsmethoden (hoofdstuk \ref{ch:ch4_light-ageing}) bevestigden voor de helft van de monsters de tekortkomingen van het principe en een gebrek aan correlatie met kleurverandering onder normale omstandigheden. Door de gegevens te interpreteren konden enkele technische en methodologische manieren worden voorgesteld om dit probleem te omzeilen, zoals het gebruik van kleurveranderingssnelheid als een meer geschikte metriek voor het toepassen van microfading data op kunstwerken in het kader van verlichtingsbeleid.\\

Globaal gezien komen alle resultaten die in dit proefschrift zijn verkregen tot één conclusie: microfadeometrie is een effectieve en efficiënte techniek voor de detectie en analyse van kleurveranderingsverschijnselen. Tegelijkertijd heeft het nog niet zijn volledige potentieel laten zien, dat naar mijn mening ligt in het vermogen om de puntsgewijze resultaten te correleren met andere analytische gegevens. Gezien het toenemende belang van beeldvormingstechnieken in de culturele erfgoedwetenschappen in de afgelopen decennia, lijkt het relevant om de microfading analyses in deze richting te oriënteren met de bedoeling om een computergestuurd bewegingssysteem te creëren waarbij aan de microfading plekken $x,y$ coördinaten kunnen worden toegekend om ze precies op objecten te lokaliseren en te documenteren. Terwijl het een verbinding mogelijk maakt van microfading resultaten met andere gegevens, zou dit ook kunnen leiden tot beeldgebaseerde voorspelling van kleurveranderingen.\\



\newpage
\textbf{\textcolor{blue}{\LARGE Résumé}}\\

\textbf{\textcolor{blue}{Dévelopement d'un nouvel appareil de micro-décoloration pour l'évaluation du risque de changement de couleur dans les oeuvres de Van Gogh}}\\

Au cours des trois dernières décennies, les restaurateurs et les conservateurs du musée Van Gogh ont recueilli de plus en plus d'indices de changements de couleur dans les peintures de Van Gogh. De nombreux projets de recherche sur les changements de couleur dans les œuvres de Van Gogh, menés depuis les années 1990, ont considérablement approfondi nos connaissances sur la question et sont passés en revue au sein du chapitre \ref{ch:ch1_general-intro}. Ainsi, les pigments sensibles à la lumière responsables de ces changements de couleur ont été identifiés et leurs mécanismes de dégradation étudiés. En outre, des expériences de vieillissement à la lumière ont été menées sur des reconstructions et plusieurs tentatives d'estimation de l'aspect original des peintures ont été réalisées. Cependant, nos connaissances actuelles sur l'échelle de temps de ce problème sont plutôt limitées. Dans quelle mesure la décoloration observée dans les peintures de Van Gogh s'est-elle stabilisée ? Et si elles se poursuivent, à quelle vitesse se produisent-elles ? En d'autres termes, peut-on s'attendre à ce que les peintures de Van Gogh subissent des changements de couleur au cours des dix prochaines années environ ? Les travaux réalisés dans le cadre de cette thèse contribuent à apporter des réponses à ces questions, en fournissant des données fiables pour l'évaluation des risques d'altération par la lumière, sur lesquelles fonder les décisions relatives à la conservation à long terme des peintures. \\

À la fin des années 1990, des chercheurs américains et britanniques ont mis au point des dispositifs permettant d'évaluer les changements de couleur induits par la lumière à l'échelle microscopique. Cette technique, connue sous le nom de \textit{micro-décoloration} ou "microfadéométrie" (\textit{microfading test} ou \textit{microfadeometry} en anglais) et décrite en détail au chapitre \ref{ch:ch2_MFT}, permet d'analyser la résistance à la lumière des objets, aidant ainsi les professionnels des musées à prendre des décisions concernant l'éclairage de leurs collections. Néanmoins, la technique présente un certain nombre de limitations qui empêchent une interprétation fiable des données. L'objectif spécifique de ce projet de doctorat est d'améliorer les analyses de micro-décoloration sur les plans technique et méthodologique afin de nous rapprocher d'une prédiction plus fiable des changements de couleur. \\

À cet égard, le projet de doctorat s'est concentré sur trois sujets liés à la micro-décoloration. Le premier visait à réaliser des progrès techniques concernant la manière dont les analyses de micro-décoloration sont effectuées (chapitre 1, section \ref{sec:stereo-MFT}). Le second avait pour but d'améliorer nos connaissances sur l'augmentation de la température à la surface des matériaux irradiés par un faisceau lumineux intense (chapitre 3, section \ref{sec:MFT-Temp}). Le troisième sujet visait à évaluer la précision des données de micro-décoloration en évaluant la validité du principe de réciprocité sur plusieurs échantillons de peinture (chapitre 3, section \ref{sec:MFT-reciprocity}) et en comparant les données de micro-décoloration avec les méthodes traditionnelles de vieillissement à la lumière (chapitre 4). \\

Les améliorations techniques réalisées ont conduit au développement d'un nouveau système appelé stéréo-MFT, qui est présenté au chapitre \ref{ch:ch3_MFT-advances}, section \ref{sec:stereo-MFT}. Ce nouveau dispositif utilise un stéréomicroscope comme élément central pour lequel le système d'optique - en combinaison avec une caméra de microscope de haute qualité - permet une meilleure caractérisation du faisceau lumineux. En fin de compte, cela améliore la précision de notre estimation de la quantité d'énergie reçue par l'échantillon, suggérant que les analyses de micro-décoloration en fonction de la dose énergétique peuvent être effectuées avec précision. L'application de l'appareil sur des échantillons de peinture et de textile nous ont en outre permis de mettre en évidence plusieurs de ses limites, qui pourraient être résolues dans le cadre de futurs projets de recherche afin d'apporter de nouvelles améliorations (chapitre \ref{ch:ch5_applications-MFT}). \\

L'utilisation d'un capteur de température infrarouge a permis de mesurer la température de surface d'échantillons de peinture irradiés par un faisceau de micro-décoloration (chapitre \ref{ch:ch3_MFT-advances}, section \ref{sec:MFT-Temp}). Les résultats ont montré une augmentation de la température significativement plus élevée que la plupart des valeurs rapportées dans la littérature. Une augmentation de la température allant de 4\unit{\degreeCelsius} à 13\unit{\degreeCelsius} a été observée pour les couleurs les plus fréquemment analysées par microfadéométrie : le jaune, l'orange et le rouge. En outre, la mise en oeuvre d'un modèle mathématique pour simuler les changements de température et le transfert de chaleur à l'intérieur des matériaux irradiés s'est avérée être un outil précieux pour comprendre l'influence de certains paramètres sur le comportement du transfert de chaleur à l'intérieur des couches de peinture. Par exemple, les paramètres liés au faisceau lumineux (puissance, taille) ou aux matériaux (conductivité thermique, capacité thermique spécifique, coefficient de transfert de chaleur, densité, etc) peuvent être ainsi étudiés.\\

\vspace{0.5mm}

La fiabilité des données de micro-décoloration est directement liée au principe de réciprocité. La recherche sur ce principe nécessite une caractérisation précise du faisceau lumineux en termes de distribution de puissance énergétique et d'exposition à la dose de lumière, soit en lx.hr, soit en MJ/m\textsuperscript{2}. Les expériences sur le principe de réciprocité (chapitre \ref{ch:ch3_MFT-advances}, section \ref{sec:MFT-reciprocity}) et la comparaison avec d'autres méthodes de vieillissement par la lumière (chapitre \ref{ch:ch4_light-ageing}) ont confirmé les échecs du principe et l'absence de corrélation avec le changement de couleur survenant dans des conditions normales pour la moitié des échantillons. L'interprétation des données nous a permis de proposer des moyens techniques et méthodologiques pour contourner ce problème, comme l'utilisation du taux de changement de couleur en tant que mesure plus appropriée pour appliquer les données de micro-décoloration aux oeuvres d'art dans le cadre des politiques d'éclairage.\\

\vspace{0.5mm}

D'une manière générale, tous les résultats obtenus dans cette thèse convergent vers une seule conclusion : la micro-décoloration est une technique efficace pour la détection et l'analyse des phénomènes de changement de couleur. En même temps, elle n'a pas encore révélé tout son potentiel, qui, à mon avis, réside dans sa capacité à corréler ses résultats avec d'autres données analytiques. Étant donné l'importance croissante des techniques d'imagerie dans la science du patrimoine culturel au cours des dernières décennies, il semble pertinent d'orienter les analyses de micro-décoloration dans cette direction avec l'intention de créer un système contrôlé par ordinateur dans lequel chaque analyse de micro-décoloration pourrait être assignées à des coordonnées $x$ et $y$ afin de les localiser précisément sur les objets. Tout en permettant de relier les résultats de la décoloration à d'autres données, ce système pourrait également permettre de prédire les changements de couleur à l'échelle de l'une oeuvre dans sa globalité.\\

\begin{figure}[!h]
\centering
\includegraphics[width=0.8\textwidth]{blank_im.png}
\end{figure}


