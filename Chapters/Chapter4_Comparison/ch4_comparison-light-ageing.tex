% !TEX root = main.tex

%%%% Title Page

\newgeometry{top=2.170cm,
            bottom=3.510cm,
            inner=2.1835cm,
            outer=2.1835cm,
            ignoremp}

\pagecolor{mygray}

\begin{titlepage}
   \begin{center}
       \vspace*{3cm}
       {\fontsize{40pt}{46pt}\selectfont \textbf{Chapter 4}}\\       
       \vspace*{3cm}
       {\fontsize{30pt}{36pt}\selectfont \textbf{Comparison} \\[1cm]
        \fontsize{30pt}{36pt}\selectfont \textbf{of} \\[1cm]
        \fontsize{30pt}{36pt}\selectfont \textbf{light ageing techniques}} \\          
   \end{center}
\end{titlepage}

\restoregeometry
\pagecolor{white}

%%%% Main text

\chapter{ Comparison of light ageing techniques}
\label{ch:ch4_light-ageing}



%%%%%%% Introduction %%%%%%%

\section{Introduction}


This chapter focuses on the comparison of three different light-ageing methods that have been applied on several oil paint-outs and blue wool samples. Although the reasons for conducting accelerated ageing experiments may vary from one project to another, there is often a desire to reproduce physico-chemical phenomena occurring under ‘normal conditions’ \sidenote{In the framework of this thesis, ‘normal’ conditions are understood as environmental museum conditions.} (\citealp[XV]{feller_accelerated_1994}; \citealp[29]{van_giesbergen_wilting_2019}), which can be viewed as reference behaviours. To validate the accelerated ageing methodology, the accelerated ageing results need to be compared against the reference data. However, this is not always possible since the reference data are often unknown to us. The experiments described in this chapter were intended to fill in this gap in our knowledge by providing light-induced colour change reference data for a set of samples. Subsequently, these reference data were compared with results obtained by microfadeometry, where the aim is to assess the validity of microfading results. In addition to this main objective, the experiments were also designed to provide preliminary answers to the following questions:

\begin{enumerate}
\item To what extent does the thickness of a paint layer influence its lightfastness?
\item To what extent does the power density of the light source influence the light-induced colour change behaviour of the samples?
\item To what extent does the spectral power distribution of the light source influence the colour change behaviour of the samples?
\end{enumerate}

Although it is generally accepted that the thicker a paint layer is, the more lightfast it is, there are no comprehensive studies to date that have established a precise relationship between the thickness of a paint layer and its light stability. The second and third questions, however, have received greater attention from the research community. The influence of the power density of light sources is an issue directly related to the reciprocity principle, previously discussed in Chapters 2 and 3. The influence of the spectral power distribution is known as wavelength dependency, for which an overview of past studies can be found in \citet[109-16]{saunders_museum_2020}. The general objective of these three questions is not so much to perform in-depth studies, but to report observations made in the context of our experiments. \\

Before getting to the heart of the matter, it is necessary to clarify some linguistic issues, by defining some terms and expressions that are often used interchangeably, but which I believe are different, such as ‘accelerated light ageing’ and ‘artificial light ageing’. The term ‘accelerated’ implies that the sample is subjected to an environment that speeds up the physico-chemical processes within the material, resulting in the alteration of its properties in a shorter period of time than when the sample is subjected to a ‘normal’ environment. Thus, the term accelerated should only be understood in relation to a previously defined standard. The term ‘artificial’ refers, in the context of light-ageing experiments, to the light source used to illuminate the sample – natural or artificial. In most cases, we use artificial light sources, but samples can also be exposed to sunlight, which is one of the methods used in this study. In this latter case, if there are no aspects of the methodology that aim to foster reactions, then the experiment can simply be defined as a natural light-ageing method.


\section{Materials and methods}

Since exposing samples to real museum conditions was not feasible due to time constraints, it was decided to implement two light-ageing systems with exposure conditions relatively close to museum specifications. Therefore, it was assumed that the results obtained with these two set-ups reasonably mimic the reality of museum exposure conditions. The first system, referred to as ‘\gls{LB}’, consisted of light boxes using LEDs currently utilised by the Van Gogh Museum in its exhibition spaces. The second set-up, referred to as '\gls{DL}', consisted of exposing samples to daylight/sunlight in front of a window of the building where the experiments were conducted\sidenote{The Studio Building (\textit{Ateliergebouw}) (Hobbemastraat 22, 1071 ZC, Amsterdam).}. The fading behaviour observed on samples illuminated with these two systems was then compared with data obtained using the stereo-MFT, for which a detailed description can be found in Chapter 3, section \ref{sec:stereo-MFT}.\\


\subsection{Light box experiments}

\subsubsection{The boxes}
\label{chap: LB_boxes}


\begin{figure*}[!h]
\centering
\includegraphics[width=\linewidth]{Chapters/Chapter4_Comparison/Figures/VGM_LED_photo-SP.png}
\caption[\hspace{0.3cm}Xicato LED used in the light boxes]{Xicato LED used in the light boxes: (a) photograph and (b) spectral power distribution.}
\label{fig:LB_VGM_LED}
\end{figure*}

The first type of ageing system consisted of custom-made wooden light boxes in which LEDs were inserted as fading light sources (Figure \ref{fig:LB_VGM_LED})\sidenote{More information about the Xicato LED can be found in Appendix \ref{app:ch4_xicato_LED}}. Two boxes, referred to as \textit{Box 1} and \textit{Box 2}, were constructed by including in each one an area where the samples are illuminated (\textit{aged area}) and an area where the samples are kept in complete darkness (\textit{dark area}). Differences in their design enable us to compare the influence of some experimental parameters (Table \ref{tab:LB_info_boxes}). Five \gls{LED} lamps were used for Box 1, which had aluminium foils applied on its inner walls (Figures \ref{fig:LB_Box1_info} and \ref{fig:LB_Box1-aged_info}) in order to maximise light reflection and therefore increase illumination levels received at the surface of the samples. Box 2 only contained one lamp surrounded by black painted walls and black textile (Figures \ref{fig:LB_Box2_info} and \ref{fig:LB_Box2-aged_info}). These differences ultimately resulted in higher light energy values inside Box 1 than Box 2 (Table \ref{tab:LB_info_boxes}). While the experimental conditions in Box 1 deviated from museum conditions, the conditions in Box 2 were designed to replicate museum lighting conditions as closely as possible\sidenote{A graphical representation of exposure conditions in the galleries of the Van Gogh Museum is showed in Appendix \ref{app:ch4_VGM_lighting}}. In order to minimise temperature rise inside the boxes, holes and wide openings were made in both boxes. In addition, one fan was positioned in each of the four corners of Box 1 to blow air towards the centre in order to help minimise the rise of temperature inside the box. The temperature and relative humidity in each box were measured regularly using iButtons\textsuperscript{\textregistered} data loggers positioned inside both areas, aged and dark\sidenote{Readings of the temperature and relative humidity levels inside the light boxes are available in Appendix \ref{app:ch4_boxes_T-RH}.}. In both boxes, minor climate variations were observed as a consequence of light exposure but their influence on the colour change behaviour of the samples was considered negligible.

\begin{figure*}
\includegraphics[width=\linewidth]{Chapters/Chapter4_Comparison/Figures/Box1_info_aged_lowres.png}
\caption[\hspace{0.3cm}Description of the illuminated area of Box 1]{Description of the illuminated area of Box 1: (a) irradiance mapping; (b) position of the samples and other elements; (c) photograph.}
\label{fig:LB_Box1-aged_info}
\end{figure*}


\begin{table*}
\caption[\hspace{0.3cm}Parameters of the light ageing boxes]{Parameters of the light ageing boxes.}
\begin{tabular}{p{5cm}p{4cm}p{4cm}}
%\begin{tabular}{@{}p{0.2\textwidth}*{3}{L{\dimexpr0.4\textwidth-2\tabcolsep\relax}}@{}}
\toprule[0.5mm]
\textbf{Parameters} & \textbf{Box1} & \textbf{Box2} \\\midrule
Light source &  \multicolumn{2}{c}{Xicato LED (Mike Stoane Lighting TTX2. 70 Mini BLE)} \\
N$^\circ$ of lamps & 5 & 1 \\
Mirror angle & \ang{45} (x1) + \ang{22.5} (x4) & \ang{45} (x1) \\
Mean illuminance (\unit{\lux}) & 8000 - 12000 & 550 - 850 \\
Mean irradiance (\unit{\watt\per\square\metre}) & 30 - 45 & 2.2 - 2.9 \\
Exposure duration (hrs per day) & 12 & 10 \\
Daily illuminance monitoring & no & yes \\
Experiment duration & 64 days & 6 to 24 months \\
Total exposure dose (\unit{\mega\lux\hour}) & 3.3 (approx.) & 5 (max) \\
Total radiant exposure (\unit{\mega\joule\per\square\metre}) & 45 (approx.) & 70 (max) \\
& & \\
Use of fans & yes (x4) & no \\
\multirow{2}{*}{Inner walls} & aged area: aluminium foil & aged area: black paint  \\
& dark area: black paint & dark area: black paint\\
& & \\
N$^\circ$ of samples & 50 & 34 \\
Position of samples & regular rotation & fixed \\
\bottomrule[0.5mm]
\end{tabular}
\label{tab:LB_info_boxes}
\end{table*}

%\begin{figure*}[!h]
%\centering
%\includegraphics[width=0.6\linewidth]{Chapter4_Comparison/Figures/VGM_lighting_conditions.png}
%\caption[\hspace{0.3cm}Lighting conditions at the Van Gogh Museum]{Representation of the lighting %conditions in the galleries of the Van Gogh Museum (Based on information provided by Domenico %Castillo (pers. comm, 10/10/2018).}
%\label{fig:VGM_lighting_conditions}
%\end{figure*}




\begin{figure*}[!h]
\centering
\includegraphics[width=\linewidth]{Chapters/Chapter4_Comparison/Figures/Box1_info_overall_lowres.png}
\caption[\hspace{0.3cm}Photos and dimensions of Box 1]{Photos and dimensions of Box 1: (a) the box covered with a black textile; (b) inside the box (aged and dark areas); (c) graphical representation of the box.}
\label{fig:LB_Box1_info}
\end{figure*}



\begin{figure*}[!h]
\centering
\includegraphics[width=0.8\linewidth]{Chapters/Chapter4_Comparison/Figures/Box2_info_overall_lowres.png}
\caption[\hspace{0.3cm}Photos and dimensions of Box 2]{Photos and dimensions of Box 2: (a) box covered with a black textile; (b) inside the box (aged and dark areas); (c) graphical representation of the box.}
\label{fig:LB_Box2_info}
\end{figure*}



\begin{figure*}[!h]
\centering
\includegraphics[width=\linewidth]{Chapters/Chapter4_Comparison/Figures/Box2_info_aged_lowres.png}
\caption[\hspace{0.3cm}Description of the illuminated area of Box 2]{Description of the illuminated area of Box 2: (a) irradiance mapping; (b) position of the samples and other elements; (c) photograph.}
\label{fig:LB_Box2-aged_info}
\end{figure*}


\subsubsection{Light exposure}


An estimation of the power density distribution inside each box was performed by interpolating several absolute irradiance measurements at precise positions (Figures \ref{fig:LB_Box1-aged_info}-a and \ref{fig:LB_Box2-aged_info}-a)\sidenote{Detailed information on these measurements is given in Appendix \ref{app:ch4_boxes_AIS}}. In Box 1, the samples were positioned on a rotating plate (Figure \ref{fig:LB_Box1-aged_info} - c). By placing the samples in a circle and rotating them by \ang{22.5} every two days, a full revolution was achieved in 32 days after which each sample had rotated by \ang{180} so a second revolution could start. Following this procedure ensured that each sample received the same amount of light dose energy. For both boxes, a digital timer (Figure \ref{fig:LB_timer}) was used to control the lamps so that they automatically switched ON and OFF at defined time, resulting in 12 and 10 hours of daily illumination per day for Box 1 and Box 2 respectively.\\

\marginpar{
\captionsetup{type=figure}
\includegraphics[width=0.8\marginparwidth]{Chapters/Chapter4_Comparison/Figures/Automatic_plug_control.JPG}
\caption[\hspace{0.3cm}Automatic plug control]{Automatic plug control.}
\label{fig:LB_timer}
}


Since the experiment duration for Box 1 was relatively short (64 days), it was assumed that the energy output of the lamps would remain stable during the experiment, which is why light intensity measurements were only performed at the beginning of the experiment. However, the fading experiment with Box 2 spanned a period of more than two years which prompted us to perform constant monitoring of the illuminance values. Consequently, a sensor from Apogee (SE-100-SS)\sidenote{An Apogee specification sheet is available in Appendix \ref{app:ch4_apogee_SE-100-SS}} was positioned in the centre of Box 2 (Figure \ref{fig:LB_Box2-aged_info} - b and c) and recorded illuminance values at 1-minute intervals (Figure \ref{fig:LB_box2_ill}). Light intensity variations in the centre of the box were then applied with the same relative amount of variations as the rest of the box, so that the energy values falling on each sample could be precisely adjusted. \\


\begin{figure*}
\centering
\includegraphics[width=\linewidth]{Chapters/Chapter4_Comparison/Figures/Box2.1_illuminance-levels.png}
\caption[\hspace{0.3cm}Average daily illuminance values at the centre of Box 2]{Average daily illuminance values at the centre of Box 2.}
\label{fig:LB_box2_ill}
\end{figure*}





\subsubsection{Samples}


A total of 89 samples were used in the framework of the light box experiments. The samples can be classified in two main groups: \gls{BWS} (Table \ref{tab:LB_info_BWS}) and \gls{PO} (Tables \ref{tab:LB_info_PO})\sidenote{Tables \ref{tab:LB_info_BWS} and \ref{tab:LB_info_PO} only show the samples that have been exposed to light. A description of the samples kept in the dark can be found in Appendix \ref{app:ch4_LB_dark_samples}}. The blue wools were purchased in 2019 from Beuth Verlag GmbH and the paint-outs were created in the laboratory of the \gls{RCE} by using a set of nine paint tubes (Table \ref{tab:paint_tubes}). The paint from these tubes, used either alone or in combination, was applied on opacity charts purchased from Leneta with the aid of metallic drawdown bars to obtain a reasonably even paint layer thickness of 50, 100, or 200\unit{\um}, as indicated on the drawdown bars\sidenote{A cross-section taken on several paint-outs showed that the thickness is usually smaller than the indication given by the drawdown bar. The thickness of the paint layer might have decreased during the drying process (see Appendix \ref{app:ch4_PO_thickness}).}. While all the paint-outs in Box 1 were made by selecting a thickness of 100\unit{\um} with the drawdown bar, the thickness of the samples in Box 2 varied (Table \ref{tab:LB_info_PO}). Five of the paint tubes (\gls{ZW1}, \gls{ZW2}, \gls{LW}, \gls{Rev-Eo-1A}, and \gls{Rev-Eo-2A}) had been made in the framework of the \gls{REVIGO} project (2014–2016) (for information on their fabrication processes see \citet{geldof_reconstructing_2018}). The lemon chrome yellow pigment, designated as \gls{CYL2b}, was prepared at the NOVA University of Lisbon with the help of Vanessa Otero according to a 19\textsuperscript{th} century recipe found in the archives of Winsor \& Newton \citep{otero_nineteenth_2017, otero_historically_2018}\sidenote{Reports on the fabrication of the two chrome yellow pigments and paint tubes can be found in Appendices \ref{app:ch4_making_CYL2b_pigments} and \ref{app:ch4_making_CYL2b_paints} respectively.}. The second chrome yellow paint (\gls{CYSig}) was prepared by mixing lead(II) chromate pigment (PbCrO\textsubscript{4}) purchased from Sigma Aldrich with custom-made linseed oil produced during the \gls{REVIGO} project \citep[3-6]{geldof_reconstructing_2018}. The paint referred to as \gls{Eo1} in Table \ref{tab:paint_tubes} was made by Alba Alvarez-Martin and Teresa Scovacricchi at the University of Antwerp according to a recipe in \citet{claro_identification_2010}\sidenote{A description of the making process is given in Appendix \ref{app:ch4_making_eosin}}, and for which eosinY pigment purchased from Sigma Aldrich was used.\\

\vspace{0.5cm}

\begin{table*}[!h]
\centering
\caption[\hspace{0.3cm}List of the paint tubes used to make the paint-outs]{List of the paint tubes used to make the paint-outs.}
\begin{tabular}{ccccC{3cm}C{4cm}}
\toprule[0.4mm]
\textbf{Id} & \textbf{Name} & \textbf{Date} & \textbf{Place} & \textbf{Pigments} & \textbf{Provenance} \\\midrule
T01 & \gls{CYL2b} & May 2019 & \multirow{5}{*}{\gls{RCE}} & Lead chromate-Lead sulfate\textsuperscript{a} & Pigment made with Vanessa Otero\textsuperscript{b} \\
T02 & \gls{CYSig} & May 2019 & & Sigma Lead (II) chromate & commercially available pigment \\ 
T03 & \acrshort{ZW1} & March 2016 & & Zinc white & \multirow{3}{4cm}{\centering Revigo Project \citep[Table 4]{geldof_reconstructing_2018}}  \\
T04 & \acrshort{ZW2} & April 2016 & & Zinc white & \\
T05 & \acrshort{LW} & Spring 2016 & & Lead white & \\ 
T07 & \acrshort{Eo1} & January 2022 & \acrshort{UA} & Sigma Eosin Y & Based on the recipe in \citet{claro_identification_2010} \\
T08 & \gls{Rev-Eo-1A} & 2014 & \multirow{2}{*}{\gls{RCE}} & Custom-made eosin & \multirow{2}{4cm}{\centering Revigo Project \citep{geldof_reconstructing_2018}}\\
T09 & \acrshort{BB} & Spring 2016 & & Bone black & \\
\bottomrule[0.4mm]
\end{tabular}
\footnotesize{\textsuperscript{a} PbCr\textsubscript{1-$x$}S\textsubscript{x}O\textsubscript{4} with $x=0.5$  ;   \textsuperscript{b} See report in Appendix \ref{app:ch4_making_CYL2b_pigments}}
\label{tab:paint_tubes}
\end{table*}


\vspace{0.5cm}

\begin{table*}[!h]
\centering
\caption[\hspace{0.3cm}Light box experiments - Info on the BWS, aged samples.]{\gls{BWS} samples used in the light box experiments: aged samples\textsuperscript{a}.}
\begin{tabular}{C{1.5cm}C{3cm}C{1cm}C{5cm}}
\toprule[0.4mm]
\textbf{Category BWS} & \textbf{Dye} & \textbf{Box} & \textbf{Sample Id} \\ \midrule
\multirow{2}{*}{BW1} & \multirow{2}{3cm}{\centering C.I. Acid blue 104} & Box 1 & 0028; 0029; 0040; 0045; 0046 \\ 
& & Box 2 & 0056 \\\hline
\multirow{2}{*}{BW2} & \multirow{2}{3cm}{\centering C.I. Acid blue 109} & Box 1 & 0031; 0032; 0041 \\ 
& & Box 2 & 0058 \\\hline
\multirow{2}{*}{BW3} & \multirow{2}{3cm}{\centering C.I. Acid blue 83} & Box 1 & 0034; 0035; 0042 \\ 
& & Box 2 & 0060 \\\hline
BW4 & C.I. Acid blue 121 & Box1 & 0037; 0038; 0043 \\
\bottomrule[0.4mm]
\end{tabular}
\footnotesize{\\ \textsuperscript{a} Information on the samples kept in the dark area is available in Appendix \ref{app:ch4_LB_dark_samples}.}
\label{tab:LB_info_BWS}
\end{table*}


\begin{table*}
\centering
\caption[\hspace{0.3cm}Light box experiments - Info on the paint-outs, aged samples
]{Paint-out samples used in the light box experiments: aged samples.
}
\begin{tabular}{C{1cm}C{2.2cm}C{1.7cm}C{2cm}C{1.7cm}C{1cm}C{1.7cm}}
\toprule[0.4mm]
\textbf{Code paint$^a$} & \textbf{Paint tubes (p1-p2)} & \textbf{Ratio (p1-p2)} & \textbf{Binding medium} & \textbf{Thickness (\unit{\um})$^b$} & \textbf{Box} & \textbf{Sample Id} \\ \midrule
\multirow{4}{1cm}{\centering P100} & \multirow{4}{2.2cm}{\centering \gls{CYL2b}} & \multirow{4}{1.7cm}{\centering 1-0} & \multirow{4}{2cm}{\centering Linseed oil} & \multirow{4}{1.7cm}{\centering 100} & \multirow{3}{1cm}{\centering Box 1} & PO003 \\ 
& & & & & & PO004 \\
& & & & & & PO005 \\
& & & & & Box 2 & PO067 \\\hline
P140.5 & \multirow{5}{2.2cm}{\centering \gls{CYL2b}-\acrshort{ZW2}} & 0.05-0.95 & \multirow{5}{2cm}{\centering Poppyseed oil} & \multirow{5}{1.7cm}{\centering 100} & Box 2 & PO072 \\
P141 & & 0.1-0.9 & & & Box 1 & PO008 \\ 
P142 & & 0.2-0.8 & & & Box 1 & PO010 \\
P145 & & 0.5-0.5 & & & Box 1 & PO015 \\
P149 & & 0.9-0.1 & & & Box 1 & PO016 \\\hline
P130.5 & \multirow{4}{2.2cm}{\centering \gls{CYL2b}-\acrshort{ZW1}} & 0.05-0.95 & \multirow{4}{2cm}{\centering Linseed oil} & \multirow{4}{1.7cm}{\centering 100} & Box 2 & PO070 \\
P131 & & 0.1-0.9 & & & Box 1 & PO028 \\ 
\multirow{2}{1cm}{\centering P135} & & \multirow{2}{2cm}{\centering 0.5-0.5} & & & Box 1 & PO031 \\
& & & & & Box 2 & PO061 \\\hline
P151 & \multirow{2}{2.2cm}{\centering \gls{CYL2b}-\acrshort{LW}} & 0.1-0.9 & \multirow{2}{2cm}{\centering Linseed oil} &  \multirow{2}{1.7cm}{\centering 100} & \multirow{2}{1cm}{\centering Box 1} & PO022 \\  
P155 & & 0.5-0.5 & & & & PO026 \\\hline
P199 & \multirow{2}{2.2cm}{\centering \gls{CYL2b}-\acrshort{BB}} & 0.9-0.1 & \multirow{2}{2cm}{\centering Linseed oil} & \multirow{2}{1.7cm}{\centering 100} & \multirow{2}{1cm}{\centering Box 1} & PO047 \\
P199.9 & & 0.99-0.01 & & & & PO049 \\\hline
\multirow{4}{1cm}{\centering P200} & \multirow{4}{2.2cm}{\centering \gls{CYSig}} & \multirow{4}{1.7cm}{\centering 1-0} & \multirow{4}{2cm}{\centering Linseed oil} & 100 & Box 1 & PO018 \\
& & & & 50 & \multirow{3}{1cm}{\centering Box2} & PO116 \\
& & & & 100 & & PO064 \\
& & & & 200 & & PO117 \\\hline
P125 & \multirow{2}{2.2cm}{\centering \gls{CYL2b}-\gls{CYSig}} & 0.5-0.5 & \multirow{2}{2cm}{\centering Linseed oil} & \multirow{2}{1.7cm}{\centering 100} & \multirow{2}{1cm}{\centering Box 1} & PO035 \\
P121 & & 0.1-0.9 & & & & PO036 \\\hline
P235 & \gls{CYL2b}-\acrshort{ZW1} & 0.5-0.5 & Linseed oil & 100 & Box 1 & PO041 \\\hline
P255 & \gls{CYL2b}-\acrshort{LW} & 0.5-0.5 & Linseed oil & 100 & Box 1 & PO043 \\\hline
P300 & \acrshort{ZW1} & \multirow{3}{1.7cm}{\centering 1-0} & Linseed oil & \multirow{3}{1.7cm}{\centering 100} & \multirow{3}{1cm}{\centering Box2} & PO058 \\
P400 & \acrshort{ZW2} & & Poppyseed oil & & & PO053 \\
P500 & \acrshort{LW} & & Linseed oil & & & PO051 \\\hline
P700 & \acrshort{Eo1} & \multirow{3}{1.7cm}{\centering 1-0} & \multirow{3}{2cm}{\centering Linseed oil} & 100 & \multirow{3}{1cm}{\centering Box2} & PO092 \\
\multirow{2}{1cm}{\centering P800} & \multirow{2}{2.2cm}{\centering \acrshort{Rev-Eo-1A}} & & & 100 & & PO096 \\
& & & & 200 & & PO124 \\
\bottomrule[0.4mm]
\end{tabular}
\footnotesize{\\ \textsuperscript{a} An explanation of the codes is given in Appendix \ref{app:ch4_code-paint}. \\ \textsuperscript{b} Thickness when the paint is wet. The paint film tends to shrink during the drying process.}
\label{tab:LB_info_PO}
\end{table*}


\newpage
\subsubsection{Measurements}

The colour of the samples was monitored by performing regular reflectance measurements with an integrated sphere spectrophotometer from Konica Minolta (model CM-2600d, illumination without \gls{UV}, small aperture SAV, specular component excluded\sidenote{Measurements with the specular component included have also been performed but not taken into account when processing the data.}) at four different positions on the paint-outs: two spots over the white background area (\gls{wbg1} and \gls{wbg2}) and two spots over the black background area (\gls{bbg1} and \gls{bbg2}) (Figure \ref{fig:LB_RS_measurements} - a). Colour measurements on the blue wool samples were performed using the same device as that used for the paint-outs but with only a white background (Leneta opacity chart) underneath the sample. A wooden frame was created for the samples to ensure that the spectrometer was placed at exactly the same position every day (Figure \ref{fig:LB_RS_measurements} - b). In addition to reflectance measurements, photographs of the samples were taken according to a precise protocol that can be found in Appendix \ref{app:ch4_methodology_photos}.\\

\begin{figure*}[!h]
\centering
\includegraphics[width=\linewidth]{Chapters/Chapter4_Comparison/Figures/RS_measurements.png}
\caption[\hspace{0.3cm}Set-up of the reflectance measurements]{Set-up of the reflectance measurements: (a) measurement spots on a paint-out ; (b) measurement on a paint-out with the Konica Minolta spectrophotometer.}
\label{fig:LB_RS_measurements}
\end{figure*}

\newpage
\subsection{Daylight experiments}
\label{sec:DL_methodology}

\subsubsection{Samples}

Similarly, for the light box experiments, the samples for the daylight experiments consisted of blue wool standards (Table \ref{tab:DL_BWS}: BW1, BW2, BW3 and BW4) purchased in 2019 from Beuth Verlag GmbH, and paint-outs. A total of 13 paint-outs were made by applying fresh oil paint on opacity charts from Leneta with the help of metallic drawdown bars (Table \ref{tab:DL_PO}). These paint-outs were made using a set of nine paint tubes (see Table \ref{tab:paint_tubes}).

\begin{table*}[!h]
\centering
\caption[\hspace{0.3cm}BWS samples used in the daylight experiments]{\gls{BWS} samples used in the daylight experiments.}
\begin{tabular}{C{1cm}C{2cm}C{3.5cm}C{2cm}C{2cm}C{1.5cm}}
\toprule[0.4mm]
\textbf{Sample set} & \textbf{Exposure period} & \textbf{Exposure duration (days)} & \textbf{Exposure dose} & \textbf{Category BWS} & \textbf{Sample Id} \\\midrule
\multirow{3}{*}{1} & \multirow{3}{2cm}{\centering \makecell{from 2021-04-17 \\ to 2021-11-15}} & \multirow{3}{*}{222} & \multirow{3}{2cm}{\makecell{17.5 \unit{\mega\lux\hour} \\ 230.2 \unit{\mega\joule\per\square\metre}}} & BW1 & 0019 \\
 & & & & BW2 & 0020 \\
 & & & & BW3 & 0021 \\\hline
 \multirow{4}{*}{2} & \multirow{4}{2cm}{\centering \makecell{from \\ 2021-11-15 \\ to 2023-02-23}} & \multirow{4}{*}{465} & \multirow{4}{2cm}{\makecell{28.1 \unit{\mega\lux\hour} \\ 444 \unit{\mega\joule\per\square\metre}}} & BW1 & 0073 \\
 & & & & BW2 & 0074 \\
 & & & & BW3 & 0075 \\
 & & & & BW3 & 0076 \\
\bottomrule[0.4mm]
\end{tabular}
\label{tab:DL_BWS}
\end{table*}



\begin{table*}[!h]
\centering
\caption[\hspace{0.3cm}Paint-out samples used in the daylight experiments]{Paint-out samples used in the daylight experiments.}
\begin{tabular}{C{1.5cm}C{2.5cm}C{2cm}C{2cm}C{1.5cm}}
\toprule[0.4mm]
\textbf{Code paint} & \textbf{Paint tubes (p1-p2)} & \textbf{Ratio (p1-p2)} & \textbf{Thickness (\unit{\um})} & \textbf{Sample Id} \\\midrule
P100 & \gls{CYL2b} & 1-0  & \multirow{4}{*}{100} & PO068 \\
P135 & \gls{CYL2b}-\gls{ZW1} & 0.5-0.5 & & PO060 \\
P130.5 & \gls{CYL2b}-\gls{ZW1} & \multirow{2}{*}{0.05-0.95} & & PO075 \\
P140.5 & \gls{CYL2b}-\gls{ZW2} & & & PO076 \\\hline
\multirow{3}{*}{P200} & \multirow{3}{*}{\gls{CYSig}} & \multirow{3}{*}{1-0} & 50 & PO111 \\
 & & & 100 & PO113 \\
 & & & 200 & PO114 \\\hline 
 P700 & \gls{Eo1} & \multirow{3}{*}{1-0} & 100 & PO094 \\
 \multirow{2}{*}{P800} & \multirow{2}{*}{\gls{Rev-Eo-1A}} & & 100 & PO098 \\
 & & & 200 & PO122 \\
\bottomrule[0.4mm]
\end{tabular}
\label{tab:DL_PO}
\end{table*}


\subsubsection{Light exposure}

The samples were positioned on a \ang{45}-inclined wooden or plastic support and exposed to daylight in front of an East-facing window inside the Studio Building (Figure \ref{fig:DL_setup} - a). Parts of the samples were covered with thick cardboard, while other parts were exposed to light. Since the paint-out samples were made on opacity charts from Leneta, one board covered paint applied over the black background while another board covered paint applied over the white background, thus resulting in four different areas (Figure \ref{fig:DL_setup} - c). \\


\begin{figure*}[!h]
\centering
\includegraphics[width=\linewidth]{Chapters/Chapter4_Comparison/Figures/window_set-up.png}
\caption[\hspace{0.3cm}Daylight experiments - Description of the set-up]{Set-up of the daylight experiments: (a) photograph ; (b) 2D graphical representation ; (c) illuminated and covered areas on paint-out samples.}
\label{fig:DL_setup}
\end{figure*}



A data logger from Elsec (765C UV) recorded illuminance values while a radiometric sensor from Apogee (SP-110-SS)\sidenote{More information about the Apogee sensor SP-110-SS is given in Appendix \ref{app:ch4_apogee_SP-110-SS} .} measured the irradiance levels (Figure \ref{fig:DL_ill-irr}): both devices had a measurement frequency of one minute. Figure \ref{fig:DL_ill-irr}-b shows that in summer, the Apogee sensor can be easily saturated, which means that the real irradiance value in that case is higher than the measured values. A solution would have been to lower the sensitivity of the sensor so that its upper limits would have been higher than 327 \unit{\watt\per\square\metre}. However, a disadvantage of this suggestion is that the sensor would have not been able to measure precise values in winter. Another possible solution would have been to estimate these values based on the illuminance values for which the sensor never reached saturation (Figure \ref{fig:DL_ill-irr} - a). However, a comparison of the illuminance and irradiance values revealed that some of the saturated irradiance values were higher than expected. The influence of this issue seemed to be relatively negligible on the final results, which is why it was decided to use the measured energy values without changing them. The illuminance and irradiance values were used to calculate the total energy doses (Figure \ref{fig:DL_light-exposure_info} - a) from which the precise value for each sample has been estimated (Table \ref{tab:DL_PO_exposure}). \\

\begin{figure*}[!h]
\centering
\includegraphics[width=\linewidth]{Chapters/Chapter4_Comparison/Figures/Window_ill-irr_Elsec-Apogee.png}
\caption[\hspace{0.3cm}Daylight experiments - Light energy values]{Daylight experiments: (a) illuminance values ; (b) irradiance values.}
\label{fig:DL_ill-irr}
\end{figure*}

\vspace{0.5cm}

\begin{table*}[!h]
\centering
\caption[\hspace{0.3cm}Daylight experiments - Light exposure information]{Daylight experiments - Light exposure information.}
\begin{tabular}{C{1.5cm}C{3.5cm}C{3.5cm}C{2.5cm}}
\toprule[0.4mm]
\textbf{Sample Id} & \textbf{Exposure period} & \textbf{Exposure duration (days)} & \textbf{Exposure dose} \\\midrule
PO060 & \multirow{4}{3.5cm}{\centering \makecell{from 2021-07-07 \\ to 2022-05-06}} & \multirow{4}{*}{303} & \multirow{4}{2.5cm}{\makecell{17.477 \unit{\mega\lux\hour} \\ 266.56 \unit{\mega\joule\per\square\metre}}} \\
PO068 & & & \\
PO075 & & & \\
PO076 & & & \\\hline
PO111 & \multirow{3}{3.5cm}{\centering \makecell{from 2022-05-30 \\ to 2022-12-05}} & \multirow{3}{*}{189} & \multirow{3}{2.5cm}{\makecell{15.655 \unit{\mega\lux\hour} \\ 251.07 \unit{\mega\joule\per\square\metre}}} \\
PO113 & & & \\
PO114 & & & \\\hline
PO094 & \multirow{2}{3.5cm}{\centering \makecell{from 2022-05-09 \\ to 2022-12-05}} & \multirow{2}{*}{210} & \multirow{2}{2.5cm}{\makecell{18.06 \unit{\mega\lux\hour} \\ 294.94 \unit{\mega\joule\per\square\metre}}} \\
PO098 & & & \\\hline
PO102 & \multirow{2}{3.5cm}{\centering \makecell{from 2022-08-29 \\ to 2023-02-23}} & \multirow{2}{*}{178} & \multirow{2}{2.5cm}{\makecell{4.87 \unit{\mega\lux\hour} \\ 76.5 \unit{\mega\joule\per\square\metre}}} \\
PO122 & & & \\
\bottomrule[0.4mm]
\end{tabular}
\label{tab:DL_PO_exposure}
\end{table*}


\newpage
Punctual absolute spectral irradiance measurements of sunlight were performed to assess the influence of the presence of clouds. These preliminary data showed that the presence of clouds influences the relative spectral power distribution of daylight radiation, especially between 500 and 800 \unit{\nm} (Figure \ref{fig:DL_light-exposure_info} - b) but whether this has any influence on the colour change of the sample is not clearly established. In this study, the samples were exposed at all times, and only the flux density values (irradiance and illuminance), without information on the spectral distribution, were recorded. \\

\vspace{1cm}

\begin{figure*}[!h]
\centering
\includegraphics[width=\linewidth]{Chapters/Chapter4_Comparison/Figures/window_dose+SP.jpg}
\caption[\hspace{0.3cm}Daylight experiments - Light exposure information]{Daylight experiments - Light exposure information: (a) Exposure dose ; (b) Spectral power distribution of daylight through glass.}
\label{fig:DL_light-exposure_info}
\end{figure*}

\vspace{0.5cm}


\subsubsection{Measurements}

The colour of the samples in the daylight experiments were monitored by performing regular reflectance measurements with an integrated sphere spectrophotometer from Konica Minolta (model CM-2600d, illumination without \gls{UV}, small aperture SAV, specular component excluded\sidenote{Measurements with the specular component included have also been performed but not taken into account when processing the data.}). For each sample, 4 random measurements per area were taken, resulting in a total of 16 measurements per sample, from which the mean and standard deviation values were calculated at each wavelength interval. The calculation from spectral to colourimetric data was done with the help of the Colour Science Python package \citep{mansencal_colour_2022} inside Jupyter notebook environments \citep{kluyver_jupyter_2016} along with the standard packages: Numpy \citep{harris_array_2020}, Pandas \citep{mckinney_data_2010}, and Matplotlib \citep{hunter_matplotlib_2007}. In addition to reflectance measurements, photographs of the samples were taken according to a precise protocol\sidenote{See Appendix \ref{app:ch4_methodology_photos}.}.\\


\subsection{Stereo-MFT}

\subsubsection{Samples}

Altogether, 12 paint-outs were analysed with the stereo-MFT. All the paint-outs were made by applying fresh oil paint on opacity charts Leneta with the aid of metallic drawdown bars (Table \ref{tab:MFT_info_PO}). These paint-outs were made using a set of nine paint tubes (see Table \ref{tab:paint_tubes}).\\

\vspace{0.7cm}

\begin{table*}[!h]
\centering
\caption[\hspace{0.3cm}Paint-out samples used in the microfading analyses]{Paint-out samples used in the microfading analyses.}
\begin{tabular}{C{1cm}C{4cm}C{1.7cm}C{2cm}C{1.5cm}}
\toprule[0.4mm]
\textbf{Code paint} & \textbf{Paint tubes$^a$ (p1-p2)} & \textbf{Ratio (p1-p2)} & \textbf{Thickness (\unit{\um})} & \textbf{Sample Id} \\\midrule
P100 & T01 (\gls{CYL2b}) & 1-0 & \multirow{4}{2cm}{\centering 100} & PO088 \\
P135 & T01-T03 (\gls{CYL2b}-\gls{ZW1}) & 0.5-0.5 &  & PO032 \\
P130.5 & T01-T03 (\gls{CYL2b}-\gls{ZW1}) & \multirow{2}{1.7cm}{\centering 0.05-0.95} &  & PO069 \\
P140.5 & T01-T03 (\gls{CYL2b}-\gls{ZW2}) & &  & PO069 \\\hline
\multirow{3}{1cm}{\centering P200} & \multirow{3}{4cm}{\centering T02 (\gls{CYSig})} & \multirow{3}{1.7cm}{\centering 1-0} & 50 & PO112 \\
& & & 100 & PO019 \\
& & & 200 & PO091 \\\hline
P300 & T03 (\gls{ZW1}) &  \multirow{3}{1.7cm}{\centering 1-0} &  \multirow{3}{2cm}{\centering 100} & PO059 \\
P400 & T04 (\gls{ZW2}) &  &  & PO056 \\
P500 & T05 (\gls{LW}) &  &  & PO054 \\\hline
P700 & T07 (\gls{Eo1}) & \multirow{2}{1.7cm}{\centering 1-0} &  \multirow{2}{2cm}{\centering 100} & PO095 \\
P800 & T08 (\gls{Rev-Eo-1A}) & & & PO099 \\
\bottomrule[0.4mm]
\end{tabular}
\footnotesize{\\ \textsuperscript{a} See Table \ref{tab:paint_tubes} for more information about the paint tubes.}
\label{tab:MFT_info_PO}
\end{table*}


\newpage
\subsubsection{Experiments}

The samples were analysed with the stereo-MFT for which a thorough description can be found in Chapter 3, section \ref{sec:stereo-MFT}. The samples were irradiated and measured according to the same experimental protocol used in the reciprocity failure experiments (Chapter 3, section \ref{sec:MFT-reciprocity}) (Figure \ref{fig:MFT-setup_2D_02} and Table \ref{tab:params-MFT}). \\

\vspace{1cm}

\begin{table*}[!h]
\centering % instead of \begin{center}
\caption[\hspace{0.3cm}Parameters of the stereo-MFT set-up]{Parameters of the stereo-MFT set-up showed in Figure \ref{fig:MFT-setup_2D_02}.}
\begin{tabular}{L{5cm}L{7cm}}
\toprule[0.4mm]
\textbf{Parameters} &  \\\cline{0-0}
Fading light source & Ocean Optics HPX-2000-HP-DUV (HPX1) \\
Filter & Linos Calflex\textsuperscript{TM}-C Heat protection (VIS-25-01) \\
Colour measurement light source & Ocean Optics HL-2000-FHSA-LL (HAL3) \\
Operational mode & Co-axial \\
Zoom & 4.0x \\
Illumination distance (\unit{\milli\metre}) & 25 \\
Collection distance (\unit{\milli\metre}) & 23 \\
Fading optical fibre & Avantes, UVIR400-ME-2 \\
Collection optical fibre & Avantes, UVIR600-ME-1 \\
Colour measurement optical fibre & Avantes, UVIR600-ME-2 \\
Illuminance (\unit{\mega\lux}) & 1.22 $\pm$ 0.063 \\
Irradiance (\unit{\watt\per\square\metre}) & 4547 $\pm$ 257 \\
Duration (min) & 45 \\
Exposure dose (\unit{\mega\lux\hour}) & 0.91 $\pm$ 0.05 \\
Radiant exposure (\unit{\mega\joule\per\square\metre}) & 12.3 $\pm$ 0.7 \\
\bottomrule[0.4mm]
\end{tabular}
\label{tab:params-MFT}
\end{table*}

\begin{figure*}
\centering
\includegraphics[width=\linewidth]{Chapters/Chapter4_Comparison/Figures/stereoMFT_2D.png}
\caption[\hspace{0.3cm}2D representation of the stereo-MFT]{2D representation of the stereo-MFT in the co-axial mode: configuration used to perform microfading analyses on samples mentioned in Table \ref{tab:MFT_info_PO}.}
\label{fig:MFT-setup_2D_02}
\end{figure*}

\newpage
\section{Results and discussion}

\subsection{Overview}

To begin with, the starting colourimetric values for each paint and for each light ageing method are shown in Figure \ref{fig:PO_CIELAB_initial}. With the exception of the eosin sample (P700), the \gls{MCDM} value for each type of paint is close or smaller than 1, which means that the initial colour of the paint-outs is relatively similar. This suggests that future colour differences greater than the \gls{MCDM} values could not really be explained by the different initial properties of the samples. Though the colour difference between similar paint-outs, \ie same code paint, is relatively small, the use of a pairwise distance algorithm applied to the intial $L^*a^*b^*$ values outlined a closer similarity between the Light box and the Daylight values than with the micro-fading values (Figure \ref{fig:pairwise_distance_Lab}). This was expected since the measurement methodology for the Light box and the Daylight experiments is similar, \ie the use of a Konical Minolta spectrophotometer with a di:8/de:8 geometry and pulsed xenon lamps with a \gls{UV} cut-off filter. In comparison, the reflectance measurements performed with the stereo-MFT differs regarding the geometry ( \ang{45}/\ang{0}) and the illumination system (halogen lamp). This affects the absolute values of the reflectance factors and colourimetric coordinates, but the amount of relative change (\dE, \dL, \da, and \db) should be comparable. \\

\begin{figure*}[!h]
\centering
\includegraphics[width=\linewidth]{Chapters/Chapter4_Comparison/Figures/Pairwise-distance_MFT-DL-LB_Lab.png}
\caption[\hspace{0.3cm}Pairwise distance computation on initial $L^*a^*b^*$ values]{Results of pairwise distance computation applied to the initial $L^*a^*b^*$ values (Dark blue colour indicates a strong similarity while yellow denotes dissimilarity).}
\label{fig:pairwise_distance_Lab}
\end{figure*}

\begin{figure*}
\centering
\includegraphics[width=\linewidth]{Chapters/Chapter4_Comparison/Figures/Comparison_DL-LB-MFT_POs_initial_CIELAB.png}
\caption[\hspace{0.3cm}Initial colourimetric coordinates for each paint]{Initial colourimetric coordinates for each paint.}
\label{fig:PO_CIELAB_initial}
\end{figure*}




\begin{figure*}
\centering
\includegraphics[width=\linewidth]{Chapters/Chapter4_Comparison/Figures/Comparison_DL-LB-MF_all_aged_dE00.png}
\caption[\hspace{0.3cm}Comparison of the \dEOO curves for each sample]{Comparison of the \dEOO curves for each sample.}
\label{fig:DL-LB-MF_all_aged_dE00}
\end{figure*}

\newpage

An overview of the results is showed in Figure \ref{fig:DL-LB-MF_all_aged_dE00}, which displays the fading curves, in terms of \dEOO unit, for all the samples according to the three light-ageing methodologies. Observations of these graphs enables us to make the following statements and interpretations:

\begin{itemize}
    \item The results of the microfading analyses do not always correlate with the daylight and light box data. This means that in some cases, microfading analyses fail to reproduce colour change occurring under normal conditions and that extrapolating the microfading results to the context of museum objects cannot be done easily. 
    \item For a majority of samples, the fading curves between the daylight and the light box experiments are relatively close. This suggests that for these samples, the influence of the spectrum of the illumination source, showed in Figure \ref{fig:lamps_rel-SP}, as long as it does not contain \gls{UV} or \gls{IR} radiation, seems to be relatively negligible. 
    \item The rate of colour change after 10 \unit{\mega\joule\per\square\metre} for the three methodologies seems to converge to a similar value. Since the rate of change over a few decades slowly decreases, we can assume a constant rate of change over a few years. This implies that the rate of colour change at the end of microfading analyses could be used as a value to predict colour change over the following years.
    \item There seems to be a recurrent pattern with the microfading data, similar to that observed when looking at the reciprocity failure experiments (Chapter 3, section \ref{sec:MFT-reciprocity}). Most microfading curves present a strong but short initial increase of the \dEOO values followed by a slow but constant rate of change. This implies that although at the beginning, the microfading \dEOO values are higher than for the light box and daylight experiments, the opposite behaviour can often be observed after 10 \unit{\mega\joule\per\square\metre}. Could the low rate of change observed after the beginning of microfading analyses be the consequence of the strong initial increase of the \dEOO values/the energy shock mentioned in Chapter 3 (section \ref{sec:MFT-reciprocity})? 
    \item Interpretation of the \dRvis values is difficult since the equation calculates the average change over the whole visible spectrum\sidenote{Figure \ref{fig:DL-LB-MF_all_aged_dE00} with \dRvis values on the $y$-axis is given in Appendix \ref{app:ch4_LB-DL-MF_all_dRvis}}. Averaging the changes over smaller and distinct wavelength bands, such as red/green/blue, help to determine where and when changes occur and has been applied to the results of BW1 sample (see section \ref{sec:RS_BW1}).    
\end{itemize}  
  
\begin{figure*}[!h]
\centering
\includegraphics[width=0.9\linewidth]{Chapters/Chapter4_Comparison/Figures/All-lamps_rel-SPD.png}
\caption[\hspace{0.3cm}Relative SPD of the light sources used in the light-ageing experiments]{Relative spectral power distribution of the fading light sources used in the light-ageing experiments.}
\label{fig:lamps_rel-SP}
\end{figure*}

\begin{figure*}[!h]
\centering
\includegraphics[width=0.9\linewidth]{Chapters/Chapter4_Comparison/Figures/Comparison_LB-DL-MFT_P100-P130.5_dLab.png}
\caption[\hspace{0.3cm}$\Delta L^*a^*b^*$ values for paint sample P100 and P130.5]{$\Delta L^*a^*b^*$ values for paint sample P100 (a - b) and P130.5 (c - d).}
\label{fig:LB-DL-MF_P100-P130.5_dLab}
\end{figure*}
    
\newpage
\begin{itemize}   
    
    \item A closer examination of the colourimetric data is required to properly compare and understand the differences between light-ageing methodologies. For instance, the lower \dEOO for the microfading curve on paint P100 is reflected in the $\Delta a^*$ and $\Delta b^*$ coordinates, where both parameters change at a lower rate than that for the daylight and light box experiments (Figure \ref{fig:LB-DL-MF_P100-P130.5_dLab} - a \& b). In some rare cases, \dE values can also be misleading. Two samples with similar \dEOO curves can behave in the CIELAB colour space in different ways. Although the \dEOO values for paint P130.5 are relatively similar from one methodology to another, the $\Delta L^*a^*b^*$ values show opposite behaviours (Figure \ref{fig:LB-DL-MF_P100-P130.5_dLab} - c \& d). This could indicate the influence of spectral power density of the light source over the fading properties of the sample, a phenomenon also known as wavelength dependency. 


    \item Figure \ref{fig:reciprocity-lines_all_dE00} combines the results from the reciprocity failure experiments (Chapter 3, section \ref{sec:MFT-reciprocity}) with the results of the daylight experiments. In most cases, the \dEOO value for the daylight experiments is higher than for the microfading analyses. The three most light-sensitive samples (PO089, BW1, PO099) present the largest differences in \dEOO values between the daylight experiments and the microfading analyses. This means that the microfading analyses fail to reproduce the colour change that occurs under normal conditions, which confirms a failure of the reciprocity principle. Many reciprocity lines resemble parabolic curves, therefore implying that low level exposures tend to produce similar colour change as high-level exposures.
\end{itemize}


\begin{figure*}
\centering
\includegraphics[width=0.9\linewidth]{Chapters/Chapter4_Comparison/Figures/Reciprocity-lines_DL-MFT_dE00-dR_subplots.png}
\caption[\hspace{0.3cm}Reciprocity lines with Daylight experiments data]{Reciprocity lines including the daylight experimental data and the microfading results: (a) \dEOO ; (b) \dRvis.}
\label{fig:reciprocity-lines_all_dE00}
\end{figure*}


\subsection{Comparison with past studies}

Two of the paint tubes used in our experiments (Table \ref{tab:paint_tubes}, T01 and T08) were fabricated according to recipes established by previous research projects and for which paint-outs have also been made and submitted to light-ageing experiments. In this subsection, the light-induced colour change results of these past research projects, mentioned in Chapter 1 (section \ref{sec:ligth-sensitivity studies}), are compared with our experimental data. \\

The first paint tube (T01) contains chrome yellow paint, fabricated with the help of Vanessa Otero according to a recipe from Winsor \& Newton archives that she labelled \textit{L2b} in her PhD dissertation. From this paint tube, sample PO.068 was made by applying the paint on an opacity chart from Leneta and exposed to daylight as described above. The light-ageing colour change data from \citet[351]{otero_historically_2018}, are displayed in Figure \ref{fig:DL-Otereo_P100_Lab} together with our data from the same paint recipe. Although the experimental methodologies between Otero’s research and our measurements differ on several aspects (\eg thickness of paint samples, presence of ultraviolet in the fading lamp, total radiant energy, spectrometer, number of measurements), which explains the difference between the \dEOO curves (Figure \ref{fig:DL-Otereo_P100_Lab}, bottom left subplot), the colourimetric coordinates show similar trends.

\begin{figure*}[!h]
\centering
\includegraphics[width=0.85\linewidth]{Chapters/Chapter4_Comparison/Figures/Otero-Patin_CYL2b_dE-CIELAB.png}
\caption[\hspace{0.3cm}Comparison of light-ageing experiments on chrome yellow samples]{Comparison of light-ageing experiments on chrome yellow samples (results taken from Otero 2018 and the daylight experiments with PO.068).}
\label{fig:DL-Otereo_P100_Lab}
\end{figure*}


The second paint tube (T.08) was fabricated in 2014 and used in the framework of the \gls{REVIGO} project to estimate the original colour of a Van Gogh painting \citep{geldof_reconstructing_2018,kirchner_digitally_2017}. Past light-ageing experiments using this paint have not been conducted, which is why direct comparison cannot be made. However, during the Red Lakes project, several eosin paint-outs were submitted to light-ageing experiments. Although the recipes from this project differ slightly differ from that used to create T.08, a comparison of the colour change data shows comparable variations of \dEOO values as well as similar trends for changes in the CIELAB colour space (Figure \ref{fig:DL-RL_P700_Lab}).


\begin{figure*}[!h]
\centering
\includegraphics[width=0.9\linewidth]{Chapters/Chapter4_Comparison/Figures/Comparison_RLproject-PO098_CIELAB.png}
\caption[\hspace{0.3cm}Comparison of the light-ageing experiments on eosin samples]{Comparison of the light-ageing experiments on eosin samples (results taken from the Red Lakes project with the permission of Klaas Jan van den Berg and from the daylight experiments).}
\label{fig:DL-RL_P700_Lab}
\end{figure*}


\subsection{Film thickness}

Results of the influence of the film’s thickness and background is shown in Figures \ref{fig:LB_thickness_dE} and \ref{fig:DL_thickness_dE}. Two types of paint (P200 and P800) were applied on opacity charts where the thickness of the paint layers varied (50, 100 and 200\unit{\um}). The two paints differ mainly in their opacity levels: while the eosin paint (P200) is highly transparent, the chrome yellow Sigma (P800) is very opaque so that even with a thickness layer of 50\unit{\um}, the background difference (\ie black or white), cannot be perceived. 


\begin{figure*}[!h]
\centering
\includegraphics[width=0.95\linewidth]{Chapters/Chapter4_Comparison/Figures/Box2.1_thickness_all-samples-legend_dE00.png}
\caption[\hspace{0.3cm}Light box samples - Film thickness]{Influence of the film thickness on the \dEOO curves for two paints in the light box experiments: (a) paint P200 ; (b) paint P800.}
\label{fig:LB_thickness_dE}
\end{figure*}


\begin{figure*}[!h]
\centering
\includegraphics[width=0.95\linewidth]{Chapters/Chapter4_Comparison/Figures/Window_thickness_all-samples-legend_dE00.png}
\caption[\hspace{0.3cm}Daylight samples - Film thickness]{Influence of the film thickness on the \dEOO curves for two paints in the daylight experiments: (a) paint P200 ; (b) paint P800.}
\label{fig:DL_thickness_dE}
\end{figure*}


The results confirmed the aforementioned statement that the thicker a paint layer the more lightfast it is. Additionally, the two paints behave differently when comparing the influence of the background. While, the \dEOO values for P800 are not really influenced by the background, the opposite behaviour occurs for P200. More precisely, the data for the eosin paint show that when the layer is sufficiently thin to allow for transparency, the background will influence the colour change behaviour of the sample. This suggests that if the chrome yellow paint were to be thinner, the background might also have had an influence on the \dEOO values. In the case of the eosin paint, the white background induces more colour changes than the black background. This could be explained hypothetically by the fact that when the light reaches white background, it is reflected, therefore increasing the interactions between light and matter. Whereas the light reaching black background is absorbed, preventing further interactions. Further research is necessary to validate or refute this hypothesis.\\

\begin{figure*}[!h]
\centering
\includegraphics[width=\linewidth]{Chapters/Chapter4_Comparison/Figures/2022-11-21_PO098-PO122_wbg_UV365.png}
\caption[\hspace{0.3cm}Photographs of eosin cross-sections]{UV-induced visible fluorescence photographs taken on two eosin paint samples exposed to daylight: (a) PO098 ; (b) PO122.}
\label{fig:photos_eosin_cross-section_DL}
\end{figure*}

The results outline the relationship between the thickness, transparency and fading properties of paint layers, especially the fading depth. Although past studies have highlighted the superficial nature of light-induced colour change \citep[39]{johnston-feller_reflections_1986}, the question of how deep a paint layer maintains its colour is still open. Photographs of UV-induced visible fluorescence of cross-section samples on two eosin paint-outs (PO098 and PO122) used in the daylight experiments revealed to be a useful technique to assess the amount of colourant along the depth of the samples (Figure \ref{fig:photos_eosin_cross-section_DL}). Additionally, the application of microspectroscopy could also be a relevant tool to monitor changes on cross-sections. 


\subsection{Light energy}

A comparison of the \dEOO curves between Box 1 and Box 2 is given in Figure \ref{fig:box1_box2_dE00}. The results for each sample show no significant differences between them, meaning that, at 45 \unit{\watt\per\square\metre}, the reciprocity principle holds for all samples. Similar results had also been found by \citet{saunders_light-induced_1996}. On a practical level, it implies that it is possible to shorten the duration of light-ageing experiments by increasing the energy of the light source at least up to approximately 45 \unit{\watt\per\square\metre} or 10 000 \unit{\lux}. \\

\begin{figure*}[!h]
\centering
\includegraphics[width=\linewidth]{Chapters/Chapter4_Comparison/Figures/Box-comparison_all-samples_dE00.png}
\caption[\hspace{0.3cm}\dEOO curves for Box 1 and Box 2]{\dEOO curves for Box 1 and Box 2.}
\label{fig:box1_box2_dE00}
\end{figure*}


\subsection{Reflectance changes on BW1}
\label{sec:RS_BW1}


This section takes a closer look at the reflectance changes of the BW1 sample that occurred in the context of the daylight experiments. Figure \ref{fig:DL_BW1_SP-dR}-a shows how the reflectance spectra varies over increasing dose exposure, for which the absolute amount of changes can be viewed as an image (Figure \ref{fig:DL_BW1_SP-dR_im} - a). Looking at the reflectance changes over the blue, green and red regions, different behaviours can be observed. While at first, the reflectance varies strongly in the red and blue regions, changes in the green only appear slowly but become predominant afterwards over changes in the blue and red regions (Figures \ref{fig:DL_BW1_SP-dR}-b and \ref{fig:DL_BW1_SP-dR_im}-b). The lack of uniformity regarding reflectance changes led us to assume that there could be several ageing processes involved at different times. If that is true, it is conceivable that microfading analyses affect these processes unevenly, ultimately leading to different colour change behaviour between microfadeometry and natural light ageing. \\

\begin{figure*}[!h]
\centering
\includegraphics[width=\linewidth]{Chapters/Chapter4_Comparison/Figures/DL_BW1_subplots-dR.png}
\caption[\hspace{0.3cm}Spectral results of the daylight experiments for BW1]{Spectral results of the daylight experiments for BW1: (a) reflectance spectra measured at different radiant exposure values ; (b) change of the reflectance values at three distinct wavelengths.}
\label{fig:DL_BW1_SP-dR}
\end{figure*}


\begin{figure*}[!h]
\centering
\includegraphics[width=\linewidth]{Chapters/Chapter4_Comparison/Figures/Window_0073_BW1_im-dRrel.png}
\caption[\hspace{0.3cm}Spectral imaging results of the daylight experiments for BW1]{Spectral results of the daylight experiments for BW1: (a) changes of reflectance values ; (b) absolute changes of the reflectance values at three distinct wavelengths.}
\label{fig:DL_BW1_SP-dR_im}
\end{figure*}


\newpage
\section{Conclusion}


In this chapter, the validity of microfading analyses has been investigated by comparing the colour change on more than 70 samples irradiated with three different light-ageing methods. Firstly, these experiences allowed us to highlight previously established knowledge on colour change phenomena. For example, the results show that in one third of the cases, microfading results do not always correlate with colour change occurring under normal conditions. This fact invites us to interpret the microfading results with caution. Additionally, not only does this study show us the importance of looking at colourimetric data beyond \dEOO values, but it also suggests the usefulness of thinking about colour changes in terms of rates rather than absolute changes. \\


In a second step, the results of these experiments provide new insights. Although there are differences in colourimetric values from one light-ageing method to another, in most cases the rate of colour change between the three methods seems to converge to a single value. This could mean that the use of colour change rates may therefore be more appropriate for applying the results of the microfading analyses to works of art in the framework of lighting policies. A comparison of the \dEOO values between the different ageing methods appears to reveal a common pattern with the microfading data (mentioned in Chapter 3, section \ref{sec:MFT-reciprocity}), where an abrupt initial rise of \dEOO values is followed by a quick decay leading to a slow but constant rate of change. The intensity of the microfading beam seems to induce a kind of ‘shock’ which provokes a sudden rise of the \dEOO values. Could this energy shock be related to the rapid decrease in the rate of colour change? In terms of engineering, would it be possible/relevant to perform the microfading analyses where the intensity of the fading beam increases over the duration of the analysis? \\

This study has touched on a number of topics in a superficial way, where I believe that further investigation would be useful. For example, the use of photographs to monitor colour changes, the development of a mathematical model to study the colour change in depth, the influence of the underlying layers on the lightfastness of the upper layers, or the characterisation of the colour change in terms of the absorption ($K$) and scattering ($S$) properties from the Kubelka-Munk model are topics that deserve to be addressed in more detail.
