% !TEX root = main.tex

%%%% Title Page

\newgeometry{top=2.170cm,
            bottom=3.510cm,
            inner=2.1835cm,
            outer=2.1835cm,
            ignoremp}
            
\pagecolor{mygray}

\begin{titlepage}
   \begin{center}
       \vspace*{3cm}
       {\fontsize{40pt}{46pt}\selectfont \textbf{Chapter 6}}\\       
       \vspace*{5cm}
       {\fontsize{30pt}{36pt}\selectfont \textbf{Conclusion}}   
   \end{center}
\end{titlepage}

\restoregeometry
\pagecolor{white}

%%%% Main text

\chapter{ Conclusion}
\label{ch:ch6_conclusion}



\section{Conclusion}


This section presents an overview of the conclusions for chapters 3, 4, and 5, as well as a general conclusion. \\


Chapter \ref{ch:ch3_MFT-advances} aimed to perform technical and methodological improvements in the field of microfadeometry. This goal was accomplished by developing a new microfading system, \ie the stereo-MFT, which uses a stereomicroscope as its central element in combination with a high-quality microscope camera. The optics of the stereomicroscope ensure better control over the fading and colour measurement spots, while the camera enables a precise characterisation of the fading beam. This device revealed a new way to perform microfading analysis by implementing a geometry that enables the fading and colour measurement processes to run independent of each other, opening up new perspectives for the field of microfadeometry. Along with the development of the stereo-MFT, two aspects of microfading analyses were investigated. The first topic focused on the rise of surface temperature during microfading analyses. The objective was to assess the rise of temperature by using an infrared temperature sensor and developing a theoretical model of heat transfer to simulate temperature changes within a material lit from above. The infrared measurements recorded temperature rises ranging from 2\unit{\degreeCelsius} to 26\unit{\degreeCelsius} maximum, depending on the energy of the light beam and the properties of the material. The second aspect focused on the reciprocity principle, adding further evidence of failure of the principle observed for a number of samples\sidenote{Failure of the principle was also observed in the experiments reported in Chapter \ref{ch:ch4_light-ageing}} and improving our understanding about reciprocity failure phenomena. Additionally, the outcome of these reciprocity failure experiments led to new technical suggestions for performing microfading analyses as well as new directions for research, so that theoretical and practical solutions could be proposed in future research projects.\\


Chapter \ref{ch:ch4_light-ageing} compared the colour change on various samples exposed to three different light-ageing methodologies (light box experiments, daylight experiments and microfadeometry). The main objective was to validate the ability of microfading analyses to reproduce light-induced colour change occurring under ‘normal conditions’, \ie museum lighting conditions. The results of these experiments showed that microfading analyses do not always correlate with the normal speed of colour change. This relates to the failure of the reciprocity principle which hinders useful interpretation and application of microfading data. Comparison of microfading results with the two other ageing techniques revealed a pattern in the microfading data which seems to suggest that the beginning of a microfading analysis is crucial for the rest of the analysis.\\


Chapter \ref{ch:ch5_applications-MFT} dealt with the application of the stereo-MFT within two projects, one on the lightfastness assessment of paintings in the framework of conservation activities and another in the context of a research project on synthetic organic pigments. While illustrating the use of the stereo-MFT in various contexts, the objective was also to confront the device with the reality of conservation and research practices and to outline the limitations of the system in the event of further technical improvements. Although the application of the device yielded useful results, several technical limitations and possible sources of error were noted. The main conclusion is that further technical and practical developments are needed to improve the application of the stereo-MFT on real museum objects, especially when transport of the device to the galleries is required. \\


Central to this PhD project was the question of whether microfadeometry could take us closer towards reliable colour change prediction, especially in the context of Van Gogh’s works. In view of the results obtained, I am confident in stating that several improvements were achieved which increased the reliability of the microfading data. On the one hand, the development of the stereo-MFT shows that a better characterisation of the microfading beam is possible, which improves the accuracy of our estimation of the amount of energy received by the sample. On the other hand, technical and methodological suggestions were proposed to bypass the failure of reciprocity, requiring further research to validate these suggestions. Although no microfading analyses have been carried out on Van Gogh's works in the framework of this thesis, the analyses carried out on paint-out samples and reconstructions provide concrete examples of what such analyses can achieve. Despite all our efforts to make historical reconstructions as accurate as possible, analysis of these objects can never give an exact estimate of the risk of colour change occurring in Van Gogh’s works. To achieve this objective, it would be necessary to carry out analyses on original works by Van Gogh. \\


Altogether,  the results obtained and conclusions drawn in this study point towards  a single, overarching conclusion: microfadeometry is an effective and efficient technique for the detection and analysis of colour change phenomena. However, it has yet to reveal its full potential, which, in my opinion, lies in its ability to extrapolate pointwise results to the surface of objects as a whole and make connections with other analytical results. Finding solutions to these two issues is fundamentally related to  imaging techniques.  \\


\section{Future perspectives}


This dissertation has contributed to the field of heritage science by exploring new methodologies to perform micro-fading analyses and improve our knowledge of light-induced colour change phenomena. On several occasions, scope for deepening of the current research or new research directions have been proposed. This section brings together these suggestions in order to guide future research projects into areas that, in the author’s opinion, seem especially relevant to advance the fields of heritage science and micro-fadeometry in particular.\\

\begin{enumerate}
    \item Establishing a clear and robust methodology to connect local micro-fading data with the global surface of objects is an important but missing element in the capability of the technique to facilitate efficient communication of its results with museum caretakers and stakeholders. There is a need to continue the research initiated by \citet{morris_virtual_2007}. In a general way,  the field of research is moving more and more towards imaging techniques and imaged-based data. The former relies on the use of scanning devices, while the later implies that whatever the techniques used, point-wise or imaging, the data should geographically be related to the object on which analyses or measurements have been performed. Such presentation of analytical data can open new perspectives regarding the interaction between different fields of knowledge and expertise, leading to a better understanding and assessment of our heritage. In other words, I believe that images can be used as another common language between all actors involved in the cultural heritage field.      

    \item To meet the afore-mentioned objective , a micro-fading system needs to be equipped with a high-quality imaging device. It is the author’s hope that the next generation of commercially available micro-fading testers will incorporate imaging systems as central elements in the \gls{MFT} design. 

    \item Further development and application of micro-fading data depends on the reliability of its predictions. As long as failure of the reciprocity principle is observed when performing micro-fading analyses, advances in the field of micro-fadeometry will be limited. A better understanding of the physical and chemical mechanisms in place when doing micro-fading analyses is crucial in our attempt to overcome the issue of reciprocity. 

    \item Creating a standardised way to process and store micro-fading files would be helpful to compare results across different institutions. In connection with the increasing and popular use of programming languages, standardisation would also help to render graphs and write reports in an efficient way. In this regard, the use of Jupyter notebooks, web-browser interfaces, and Latex documents seem relevant tools that should be used more often than is now the case. 

    \item There is a lack of experimental data and knowledge about the depth of light-induced colour change. Observing cross-sections of faded eosin paint-outs under \gls{UV} enabled us to perceive colour gradients through the depth of the samples\sidenote{See figure \ref{fig:photos_eosin_cross-section_DL}}. With the development of micro-spectrometry, the application of this technique for studying in-depth colour change could be very useful. 

    \item The Kubelka-Munk two flux model has been widely applied to predict the reflectance of paint layers. However, modelling of light-induced colour change in terms of the absorption ($K$) and scattering ($S$) properties given by the Kubelka-Munk theory has never been implemented. To what extent could the application of the Kubelka-Munk theory for the modelling of light-induced colour change be useful? Could it provide  a theoretical framework to create digital predictions of light-induced colour change of works of art? All the paint-outs used in this dissertation have been applied on a white \& black Leneta chart, which leaves the possibility - when the paint is not fully opaque - to calculate the $K\&S$ values.

    \item The vitality of lighting technologies, especially LEDs, and their application in the context of museum exhibitions forces the heritage community, especially in conservation and research departments, to continuously study and assess the impact of these new technologies on the preservation of cultural objects.
    
\end{enumerate}












