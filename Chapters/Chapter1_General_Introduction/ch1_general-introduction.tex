% !TEX root = main.tex

\newcommand{\HRule}{\rule{\linewidth}{0.5mm}} 	% horizontal line and its thickness

%%%% Title Page

\newgeometry{top=2.170cm,
            bottom=3.510cm,
            inner=2.1835cm,
            outer=2.1835cm,
            ignoremp}

\pagecolor{mygray}

\begin{titlepage}
   \begin{center}
       \vspace*{3cm}
       {\fontsize{40pt}{46pt}\selectfont \textbf{Chapter 1}}\\       
       \vspace*{3cm}
       {\fontsize{30pt}{36pt}\selectfont \textbf{General} \\[1cm] 
        \fontsize{30pt}{36pt}\selectfont \textbf{introduction}} \\           
   \end{center}
\end{titlepage}


\pagecolor{white}
\restoregeometry


\chapter{ General introduction}
\label{ch:ch1_general-intro}

This general introduction is divided into four main sections. It starts with a short explanation of light-induced colour change on cultural heritage objects as the central topic of this dissertation, going on to describe the latter in the next section. A third section on the challenge of light management in cultural heritage institutions gives the reader the ground knowledge necessary to connect the outcomes of this project to the practice of conservation in institutions. Finally, in order to place this research in a historical context, the last section provides an overview of past studies on colour changes in Van Gogh's works. \\



%%%%%%% The phenomenon of light-induced colour change on cultural heritage objects %%%%%%%

\section{The phenomenon of light-induced colour change on cultural heritage objects}


From a materials perspective, a colour change consists of a shift in colour perception of a surface over time. Although scientific tools and methods have been developed to mathematically express colours and colour changes in a precise and objective way, it is important to remember that colour and colour change are perceptual phenomena. This means that the appearance and interpretation of a colour or a colour shift also depends on the response of the observer from a physiological, psychological and cultural point of view. In other words, we can quantify a colour change in an objective way, but whether or not we find it to be disturbing or even consider it as damage  cannot be determined with exact science tools, as Salvador Mu\~noz Vi\~nas noted: 

\textit{It is very interesting to note that neither intention nor value are material, scientifically determinable factors, so that ultimately the all-important notion of damage is neither within the reach of science, nor is it a property of the object, but the result of a personal subjective judgement.} \citep[102]{munoz_vinas_contemporary_2005}. \\


\marginpar{
\captionsetup{type=figure}
\includegraphics[width=4cm]{Chapters/Chapter1_General_Introduction/Figures/Materials_values_1.png}
\caption[\hspace{0.3cm}Values-materials relationship]{Values-materials relationship.}
\label{fig:values_materials}
}


Cultural heritage objects are no more than the sum of various materials assembled together to which society attributes some specific values, making a difference between the bicycle wheel in your garage and the one displayed at the Centre Pompidou in Paris – although both wheels might be virtually identical from a material point of view. The interdependency between values and materials is a key aspect of cultural heritage artefacts (Figure \ref{fig:values_materials}). As a visitor, the interaction with these objects is initially visual in most cases, allowing access to the values of the objects. In conservation, we begin by establishing the values of the objects, so that our actions on the materials can influence the values appropriately. In other words, the materials of cultural heritage objects form both a carrier and a gateway to the values and information they convey. Environmental agents can directly affect materials and modify their aesthetics, which ultimately can cause a change in the values and meanings conveyed to the observer. For example, the ink on a letter can vanish hampering our ability to read its content, or the highlights on a painting may become less visible thereby modifying the effect of tonal contrast.\\


Among the many factors causing colour change on materials, exposure to optical radiation is responsible for the majority of colour change occurring on objects\sidenote{Exposure to optical radiation may also weaken the physical structure of objects through cracking and embrittlement \citep[p.126]{saunders_museum_2020}.}. Unlike other environmental agents – such as temperature, relative humidity and, oxygen concentration – light is the only one that cannot be removed without preventing visual access to the object. Any use of the object by a person requires it to be illuminated, which can potentially result in colour changes. \\


\begin{figure}[!h]
\centering
\includegraphics[width=0.8\textwidth]{Chapters/Chapter1_General_Introduction/Figures/colour_description_1.png}
\caption[\hspace{0.3cm}Description of colours]{Description of colours.}
\label{fig:colours_description}
\end{figure}


There are three main ways of describing colours and colour changes: verbally, visually and numerically (Figure \ref{fig:colours_description}). Although an abundant vocabulary gives us the possibility to convey a fair estimation of colour, in the context of a scientific research project, words lack precision and objectivity\sidenote{Limitations of language in the field of conservation have also been outlined by \citet{bucklow_description_1997} when studying craquelure patterns on paintings.}. This is why, in the framework of this PhD project, visual and numerical description of colours will mainly be used. From a colorimetric perspective, variations within a colour can be divided into three types of change, which are themselves derived from colourimetric characterisation of colours. A colour can vary in lightness ($L^*$  in the CIELAB colour space)\sidenote{More information about the \gls{CIE} colour spaces can be found in Chapter 2, section \ref{sec:colour_measurement}.}, becoming darker or lighter. It can also alter in saturation ($C^*$ metric); usually a colour becomes less saturated upon light exposure, thus appearing less intense or greyer. Finally, and more rarely seen, the hue of a colour can shift ($h$ metric). Such change is usually observed for thermochromic substances which undergo a reversible change of colour when heated or cooled. \\


Although the risk of colour change due to light exposure is higher than for any other risk factor causing colour change, it does not mean that the risks for museum objects today are higher than they were in the past. Over the last 25 years, the number of exhibited objects has not noticeably increased\sidenote{\url{https://www.egmus.eu/nc/en/statistics/complete_data/} (accessed 15/07/2021).} and the light intensity levels for exhibition have even tended to decrease. But our practices of conservation and our perception of colour change phenomena have evolved. Studies on colour changes conducted since the late 1990s have increased awareness among cultural heritage professionals\sidenote{A review of past studies in colour changes in Van Gogh is provided in section \ref{sec:overview_past_studies}} as well as the general audience\sidenote{Articles in mainstream newspapers help to bring the topic into the public space. An illustrative list of articles is given in Appendix \ref{app:ch1_articles_lid}}. In addition, the development of preventive conservation led to the need for more accurate predictive data in support of decision-making for collection care strategies. Moreover, lighting systems have evolved considerably over the last decades, with more frequent use made of \gls{LED} light sources in exhibition spaces. The dynamism of this latter technology forces the cultural heritage community to perform their own research in order to assess and validate the use of LEDs for the illumination of cultural objects \citep{michalski_led_2020}. All in all, the motivation to pursue this research topic in the form of a PhD was created by new \gls{LED} lighting installations in the exhibition spaces of the Van Gogh Museum\sidenote{In 2015–2016, the Van Gogh Museum replaced halogen light sources with \gls{LED} lamps: Domenico Casillo, light designer at the Van Gogh Museum, personal communication, 10/10/2018). } and increased awareness of the impact of light-induced colour change based on the research performed over the last decades on the Van Gogh Museum collections.\\



%%%%%%% About this thesis %%%%%%%

\newpage
\section{About this thesis}

\subsection{Subject}

\marginpar{
\captionsetup{type=figure}
\includegraphics[width=\marginparwidth]{Chapters/Chapter1_General_Introduction/Figures/PhD_intersection_fields.png}
\caption[\hspace{0.3cm}Fields of knowledge within the PhD project]{Fields of knowledge within the PhD.}
\label{fig:knowledge_fields}
}

The subject of this PhD lies at the intersection of three distinct fields of knowledge (Figure \ref{fig:knowledge_fields}). The first relates to the technical development of tools for studying the lightfastness of materials, specifically those used in works of art. Since the last decade of the 20\textsuperscript{th} century, our ability to assess the stability of coloured materials exposed to light has improved through the development of \gls{MFT} \citep{pretzel_determining_2000, whitmore_predicting_1999}. The second field concerns paint and colour technology, which is essential when attempting to understand the materiality of works of art. For example, 'historically accurate’\sidenote{There is an ongoing debate regarding use of the term 'historically accurate', other terms such as 'historically influenced, appropriate or informed' have been proposed instead \citep{carlyle_reconstructions_2020}.} paint reconstructions used in several research projects over the last two decades have proved to be an important element for interpreting phenomena observed on actual objects \citep{carlyle_historically_2007}. The last field of knowledge discussed in this thesis deals with the study of colour and colour changes in Van Gogh's paintings and works on paper. It is a known fact that some of the colours used by Van Gogh change upon light exposure and over the past three decades, conservators and curators have gathered increasing evidence of colour changes having occurred in Van Gogh’s paintings \citep{hendriks_paintings_2016} as well as in drawings (\citealp{meedendorp_hand--hand_2013,shelley_technical_2005}; \citealp[428]{vellekoop_van_2013}). While past studies on colour changes in Van Gogh’s works have greatly deepened our knowledge on the issue, they also raised more questions and opened up new avenues for conservation practice that have yet to be explored. This doctoral research aims to take a next step by combining knowledge from the three aforementioned fields linked to the study of light-induced colour change and focusing on the development and application of an improved microfading methodology to study colour changes in Van Gogh's works. \\

\subsection{Research questions and objectives}



Although, the light-sensitive pigments responsible for the colour changes occurring on Van Gogh's works have been identified and their degradation mechanisms investigated \citep{bommel_investigation_2005, burnstock_comparison_2005, monico_degradation_2013-1}, our actual knowledge of the timescale of this problem is rather limited. For example, to what extent has the observed fading and discolouration in Van Gogh’s paintings now stabilised? And if still ongoing, how fast does it occur – can we expect to see colour changes taking place in Van Gogh's artworks over the next ten or twenty years? Although answering these questions lies beyond the scope of this thesis, they were crucial for determining the direction of this work. To provide answers requires valid data on which to base decisions regarding the long-term preservation of the paintings. Supplying the museums with reliable data for light-fading risk assessment is the ultimate goal of this research (Figure \ref{fig:PhD_objectives}, yellow circle). Centred around this main objective, the project is structured in three main parts, each with its own goals (Figures \ref{fig:PhD_objectives} and \ref{fig:PhD_work_packages}, green rectangles). The first part aims to improve microfadeometry by implementing a new device along with its methodology. The purpose of the second part is to increase our knowledge and understanding of 19\textsuperscript{th} century paint materials as well as colour change phenomena. Once these two objectives have been met, a third part of the project will entail the application of microfadeometry to a variety of samples for which the light exposure risk can be modelled. \\

\begin{figure*} %[!h]
\centering
\includegraphics[width=\linewidth]{Chapters/Chapter1_General_Introduction/Figures/PhD_objectives.png}
\caption[\hspace{0.3cm}Objectives of the PhD]{Objectives of the PhD.}
\label{fig:PhD_objectives}
\end{figure*}

The three main parts of the project also aim for some longer-term goals (Figure \ref{fig:PhD_objectives}, blue rectangles). Firstly, through the communication of research outcomes as well as collaborative projects with other institutions, I hope that the improvements accomplished within the field of microfadeometry will carry forward into the conservation, scientific and industrial communities. For example, some of the technical improvements may be implemented by industrial partners for the next generation of microfading devices. Similarly, the knowledge acquired on colour changes observed in 19\textsuperscript{th} century paint materials in the second part of the project can be used for improved digital visualisations of artworks providing useful tools to solicit viewers’ opinion on perceived changes. Lastly, the provision of reliable microfading data will benefit the long-term conservation of artworks when utilised within the framework of a lighting policy.


\subsection{Structure and methodology}

In accordance with the previously described fields of knowledge and PhD objectives, the PhD work programme is structured in three main parts (Figure \ref{fig:PhD_work_packages}, green rectangles). Several interdependent sub-research projects (light blue rectangles) bind together the three main strands of research. For example, projects 1.3, 2.2, 2.3 and 3.1 all use model paint-outs that have been created in project 2.1. Chapter 3 of this thesis draws together all the projects from Part 1, while Chapter 4 compares the results of different light-ageing techniques therefore including sub-projects from parts 2 and 3, specifically 2.2, 2.3 and 3.1. The application of microfadeometry to museum objects is discussed briefly in Chapter 5. \\

The following subsections aim to introduce the types of sample and analytical methods employed for this PhD project. More detailed information about sample preparation and analytical parameters is given within the section of each research project.\\

\begin{figure} %[!h]
\centering
\includegraphics[width=0.9\linewidth]{Chapters/Chapter1_General_Introduction/Figures/PhD_working_structure.png}
\caption[\hspace{0.3cm}Structure of the PhD thesis]{Structure of the PhD.}
\label{fig:PhD_work_packages}
\end{figure}

\subsubsection{Objects of investigation}

A variety of objects has been used throughout this PhD. This section gives a clear terminology for each of them.

\begin{itemize}
    \item \textbf{Reference samples}: Samples used internationally as references in the field of colourimetry and microfadeometry. These are typically Pantone, Macbeth ColorChecker, or BS381C \citep{british_standards_institution_bs_1996} colour charts and \gls{BWS} (DIN EN ISO 105-B02, \citep{deutsches_institut_fur_normung_ev_din_2001}). 	
    \item \textbf{Model \gls{PO}/reconstructions}: Contemporary paint layers reconstructed according to the physical and chemical properties of Van Gogh paintings. They are not paint layers from actual Van Gogh paintings but historically informed reproductions.
    \item \textbf{Cross-section samples}: Paint samples, taken from model paint-outs or from an original work of art, embedded in polyester resin and prepared as cross-sections.
\end{itemize}
 
\subsubsection{Analytical methods}

Throughout this PhD project, visible reflectance spectroscopy was the main technique of investigation employed. It was used either with the help of a photospectrometer, such as the CM-2600d from Konica Minolta, or as a self-assembled system in which a spectrometer was combined with a light source. The latter option was implemented in the microfading device developed in this project (Chapter 3, section \ref{sec:stereo-MFT}). Other devices were also used, mainly to monitor the environmental conditions such as the light energy (in photometric and radiometric units), temperature or relative humidity and are mentioned in detail within their corresponding sections.\\


\subsubsection{Data processing methods}


The processing of every data acquired during this PhD followed three main and general rules:

\begin{itemize}
\item \textbf{The use of simple and open-source file format.} The durability of the scientist's work often depends on ensuring long-term access to the research results. Consequently, as elaborated in Chapter \ref{ch:ch3_MFT-advances}, page \hyperlink{page.136}{136}, it was chosen to use open file format for this study, seen as the best way to ensure that files can continue to be opened in the future.  

\item \textbf{The use of programming languages.} The data generated by the analysis of samples and works of art is becoming increasingly important in terms of quantity and size. In other words, the number of analyses increases as well as the amount of information for each measurement. Additionally, new techniques and devices are regularly developed adding to the increasing amount of data. Faced with this mass of data, I believe that it is relevant to use IT tools that enable the data to be processed quickly and efficiently. Although traditional and popular data processing and visualisation software, such as the Microsoft Office programs, offer certain significant advantages, it is clear that they are not suited to repeated, automatic processing of data. Hence the need to use programming languages to process and visualise data. There is a large number of computer languages, each with their own advantages and disadvantages depending on the context of use. The Python language seems well suited to the cultural heritage field and compared to other languages is fairly easy to learn and thus accessible to most scientists. It also offers a wide range of tools for data processing and visualisation which is particularly relevant when combining imaging and point-wise data.

\item \textbf{The use of a unique digital platform.} Most of the time, various observational and analytical techniques are used to study or characterise the materials used in works of art. Visualising the results and data acquired with each tool often requires the use of specific software developed by the company that manufactured or sold the device. In other words, there are as many digital platforms as there are analysis devices, which makes it difficult to relate the results of different types of analysis. In the framework of this PhD project, a unique digital platform has been used to process and visualise the data acquired from various devices. This digital environment, called \textit{Jupyter}\sidenote{See the Jupyter website for more information: \url{https://jupyter.org/}}, enables the use of programming languages such as Python and offers the possibility to open  various types of files and formats (images, text, numerical data, pdf, etc.), which makes the Jupyter ecosystem very versatile and flexible. 
\end{itemize}




%%%%%%% Challenges of light management in cultural heritage institutions %%%%%%%

\newpage
\section{Challenges of light management in cultural heritage institutions}

\vspace{0.3cm}

\begin{figure*}[!h]
\centering
\includegraphics[width=0.88\textwidth]{Chapters/Chapter1_General_Introduction/Figures/Risk_management_2.png}
\caption[\hspace{0.3cm}Risk management context applied to lighting policy]{Risk management context applied to lighting policy.}
\label{fig:risk_management}
\end{figure*}

\vspace{0.3cm}

This section looks at colour change as seen from a conservation perspective within the framework of cultural institutions. The purpose is to study how museums respond to the phenomenon of colour change. In other words, how do cultural institutions deal with this issue and what kinds of methodology and tools can they utilise to face up to this challenge? \\ 

More broadly, the issue can be approached from a risk management perspective for which standard protocols \citep{international_organization_for_standardization_isotc_262_iso_2018} and procedures have been developed, such as the ABC method \citep{michalski_abc_2016} and QuiskScan \citep{brokerhof_quiskscanquick_2016}\sidenote{For a good introduction to the topic of risk management applied to cultural heritage objects see \citet[9-16]{brokerhof_risk_2017}.}. Addressing light-induced colour change from a risk management perspective can be akin to implementing a lighting policy framework which contains the risk management steps\sidenote{Section \ref{sec:lighting policy framework} focuses on the risk management steps.}. These steps can be applied from either a category-based or an object-based perspective (Figure \ref{fig:risk_management}). Although both approaches follow risk management procedures, the two methods of application differ.  \\ 

In the category-based approach, the lighting policy is defined as a set of written recommendations or rules guiding the illumination of cultural objects, hence the common use of the term illumination guidelines\sidenote{It is interesting to note that most guidelines focus on illumination levels in an exhibition context, ignoring exposure during conservation treatments. To my knowledge, Garry Thomson’s The Museum Environment is one of the rare publications to discuss it \citep[44]{thomson_museum_1978}.}. Many institutions have produced such documents in which the main information is displayed in a table classifying materials according to their light sensitivity, with guideline light exposure levels and duration given for each category (\citealp[Tables 1 and 2]{ashley-smith_continuing_2002}; \citealp[10–11, Tables 2 and 3]{instituut_collectie_nederland_het_2005})\sidenote{For a good review of existing guidelines up to 2010 see \citet[28-35]{del_hoyo-melendez_study_2010}. While minor developments have occurred since 2010, nothing has changed radically since that date so the information and comments provided by Julio del Hoyo-Meléndez are still valid.}. Although the number of light-sensitivity categories and recommended illuminance values might slightly differ from one institution to the next, most classify the objects in three categories of light sensitivity – high, moderate, low – with, respectively, their corresponding recommended illuminance range values – 50, 75–150, 200 lux or above; which is quite similar to the categories defined by \citet[23]{thomson_museum_1978}. The lighting decision for each object is defined by the category assigned to it. In other words, the approach starts from a generic group and then transfers the properties relating to that group to individual objects. The method is relatively easy and fast to implement since it does not require analyses or measurements on the objects. \\ 



Conversely, the object-based approach starts with analysis of the object, providing data that can be used to make decisions leading to specific actions for each object. When several objects have similar characteristics, the original data can be extrapolated to a collection of objects. Contrary to the previous approach, this one goes from a single object to a group of similar items. However, since it requires more time, money, and knowledge than the first approach, only larger institutions have access to such resources. \\

The following subsections explore the structure of the lighting policy framework in detail. The first starts with a description of the general context, defining the actors in play and their relationships. A second subsection poses the problem of the light dilemma which is directly related to colour change phenomena. And lastly, in a third subsection, the methodology and tools available to cultural institutions in order to confront this matter are reviewed. Since very comprehensive and up-to-date publications on the aspect of light management by cultural heritage institutions are available, such as \citet{michalski_light_2018} and \citet{saunders_museum_2020}, I will refer to the state-of-the-art knowledge they present rather than repeating this information in this dissertation.\\

\subsection{Institutional context}

Although in 2022, the updated definition of a museum given by the \gls{ICOM} mentions several objectives\sidenote{“A museum is a not-for-profit, permanent institution in the service of society that researches, collects, conserves, interprets and exhibits tangible and intangible heritage. Open to the public, accessible and inclusive, museums foster diversity and sustainability. They operate and communicate ethically, professionally and with the participation of communities, offering varied experiences for education, enjoyment, reflection and knowledge sharing.” (ICOM definition adopted on August 24\textsuperscript{th} 2022)
(\url{https://icom.museum/en/resources/standards-guidelines/museum-definition/} , accessed on 05/01/2023).}, globally speaking, the objectives of cultural institutions can be summed up in just two words: preservation and dissemination. The primary mission of these institutions is the preservation of cultural heritage, first by collecting objects of interest and then by preserving their material integrity through conservation measures while enhancing their associated meanings and values through studies and research activities from a conservation, curatorial and scientific point of view. By dissemination, I mean all activities involving the collection and seeking for communication with external parties. It includes a broad range of activities such as sharing knowledge related to the objects and their makers, encouraging others to learn about the collection, inspiring people, supporting research projects on the collection, and is carried out through publications as well as through the organisation of periodic or permanent events, ranging from group workshops to temporary exhibitions on a specific subject.\\

Exhibiting objects, works of art or not, always involves four main actors\sidenote{The exhibition context can be approached through the \gls{ANT}, hence the use of the word actor to refer to both people and objects. More information about \gls{ANT} can be found on \url{https://en.wikipedia.org/wiki/Actor\%E2\%80\%93network_theory}, accessed on 06/01/2023.}: the item itself, a light source to be able to see the item, an observer that acknowledges the object's presence and a setting such as an institution that links the aforementioned actors in a single context (Figure \ref{fig:institution_exhibition} - a). Each actor within the exhibition ecosystem has a set of parameters (Figure \ref{fig:institution_exhibition} - b, italic text) that will need to be characterised by the institution. It is also important for the institution to fully and properly control the light source in order to optimise the relationship between the observer and the object: the better the viewing conditions, the more details and nuances the observer can perceive, therefore enabling him or her to fully access the values of the object. By controlling the object's exposure through management of the lighting system, the museum can partly deal with the colour change phenomena, but will inevitably be  confronted with the  dilemma  described in the following paragraphs.\\

\begin{figure*} %[!h]
\centering
\includegraphics[width=0.95\linewidth]{Chapters/Chapter1_General_Introduction/Figures/Institution_exhibition.png}
\caption[\hspace{0.3cm}The cultural institution context]{The cultural institution context: Actors involved in the exhibition of objects (a) ; Ecosystem of exhibition event (b).}
\label{fig:institution_exhibition}
\end{figure*}


\subsection{Light dilemma: saving versus seeing}

Light possesses the ambivalent capability to reveal the beauty of objects to an observer while simultaneously contributing to their decay. For some works of art – mostly those containing pigments and dyes – the optimal conservation environment would be to leave the object in complete darkness\sidenote{Some colour changes are known to take place in the dark, \eg darkening of oil binder in paint layers, which is a reversible process.}, which obviously prevents its visual appreciation by the observer. As \citet{henderson_beyond_2020} noted, the preservation of our cultural heritage for as long as possible should not be at the expense of present generations. This conflict of interest between the preservation of the item (\textit{saving}) and its display (\textit{seeing}) is designated as the light dilemma with which every cultural heritage institution that exhibits its collection is confronted (Figure \ref{fig:lighting_policy_framework}). Fully satisfying both sides of the equation is impossible, meaning that we cannot expect to exhibit a light-sensitive object under optimal viewing conditions – that is with no restrictions on the light intensity and exposure duration – without colour change taking place over time. Achieving a balance between the two therefore offers a compromise solution to the light dilemma. Two types of response have been developed to accomplish this solution: a category-based and a risk-based approach. The traditional category-based approach, mainly established by \citet{thomson_museum_1990}, divides materials or objects according to several categories of colour fastness and has provided the basis for the majority of lighting policies developed until now. However, in the context of this dissertation, I would like to place a greater focus on the second approach as a more appropriate way to strike a balance between saving and seeing, as will be explained. 


\subsection{Lighting policy framework}
\label{sec:lighting policy framework}

This subsection describes the lighting policy framework which consists of several steps to be undertaken in the context of risk management. As already mentioned, several different methods have been developed, but all include – with some degree of variation – the five steps described in the ISO guideline on risk management \citep{international_organization_for_standardization_isotc_262_iso_2018}:


\marginpar{
\captionsetup{type=figure}
\includegraphics[width=\marginparwidth]{Chapters/Chapter1_General_Introduction/Figures/Van_Driel_2018_Thesis-triangle.png}
\caption[\hspace{0.3cm}Risk management triangle]{Risk management triangle \citep[16]{van_driel_white_2018}.}
\label{fig:risk_management_triangle}
}


\begin{enumerate}
    \item Establish context.
    \item Identify risks.
    \item Analyse risks.
    \item Evaluate risks.
    \item Reduce risks.    
\end{enumerate}

These steps have been grouped in the risk management triangle (Figure \ref{fig:risk_management_triangle}) designed by Birgit van Driel and used as the methodology for her PhD  \citep[16]{van_driel_white_2018}. Here the idea is to apply this same methodology to the light dilemma, supporting the development of a lighting policy framework that ensures that a balance between the colour change phenomena (saving) and access to the object's values (seeing) is achieved (Figure \ref{fig:lighting_policy_framework}). The method’s first step is to characterise and understand both aspects; on the saving side, the objective is to gather reliable information that will help to minimise colour change, while on the other side, the aim is to maximise access to the values of the object. This initial phase often costs much time and energy but constitutes the important basis from which museum professionals can make informed decisions regarding the lighting of objects. In a next step, the gathered information is related to a time scale in order to make predictions about future change represented in the form of digital visualisations that can help institutions to define a preservation target for their collections. The last step deals with actions that can be taken to reduce/prevent risks, such as the choice of appropriate illuminance values and exposure durations (lighting exposure guidelines) and the monitoring of light levels and colours of objects on display. \\


\begin{figure*}[!h]
\centering
\includegraphics[width=\linewidth]{Chapters/Chapter1_General_Introduction/Figures/Lighting_policy_framework_steps.png}
\caption[\hspace{0.3cm}Lighting policy framework in a risk-based methodology]{Lighting policy framework in a risk-based methodology.}
\label{fig:lighting_policy_framework}
\end{figure*}

\newpage
\subsubsection{Characterise and understand}

The side of the balance dealing with minimising colour change involves three main types of information listed below:
\begin{itemize}
    \item Object values and colour-value function 
    \item Material composition
    \item Light sensitivity assessment (vulnerability)\sidenote{This is where micro-fadeometry can help to assess the light sensitivity. A section about micro-fadeometry in the context of light management is given in Chapter 2, section \ref{sec:MFT_heritage_field}}
\end{itemize}
	
Though the importance of assigning values to cultural objects (specifically monuments) had already been recognised by the Austrian art historian Alois Riegl (1858-1905) in his work \textit{Der moderne Denkmalkultus} \citep{riegl_moderne_2011}\sidenote{The book was first published in 1903.}, it took nearly a century before the notion of value-based decision-making entered conservation on a broader basis \citep{munoz_vinas_ethics_2020}. Dutch art historian and conservation philosopher, Ernst van de Wetering (1938-2021), put forward that all conservation decisions involve compromise as they are based on the outcome of negotiating different, usually conflicting values of objects among stakeholder groups. Around 1980, he represented these conflicting values as opposing forces that cannot be easily reconciled in the form of a circular decision-model for conservation \citep{van_de_wetering_roaming_1987}, which subsequently became incorporated in an expanded decision-model for contemporary artworks \citep{foundation_for_the_conservation_of_modern_art_decision-making_1999}, recently revised during a workshop organised by the Cologne Institute for Conservation Science \citep{giebeler_decision-making_2019}. With the development of preventive conservation, methodologies have been provided to help care-takers assess the values of their collections \citep{avrami_values_2019, avrami_values_2000, cultural_heritage_agency_assessing_2014, de_la_torre_assessing_2002, reed_reviewing_2018, russell_significance_2009}. In the context of colour change, the objective is to characterise the importance of colours for the significance of the object and how its values are affected by colour changes. Among the set of values assigned to an object or a collection, colour changes usually affect only a part of these values, hence the need to define which values are at risk \citep[63-65]{beltran_microfading_2021}. The colour-value function relates colour changes to variations in the values at risk (Figure \ref{fig:characterisation_functions} - a). The object's values and the colour-value function are perhaps the more complex parameters to identify as they cannot be measured by scientific tools and are largely dependent on the social context in which the object is perceived \citep{taylor_representation_2008}. The desired outcome should ideally take into account the point of view of all parties related to the object, ranging from curatorial staff to museum visitors\sidenote{Step 1 of the \gls{SBMK} model establishes what the mode of decision-making will be \citep[5]{giebeler_decision-making_2019}. Since different stakeholders may be given different degrees of involvement/power, the desired decision is not necessarily a consensus, it may be taken by an individual, by a majority vote, etc.}. Reaching this objective requires continual dialogue and discussion between all sides. The identification of the object's constituent materials usually requires analyses, either in-situ measurements and/or chemical analyses on samples taken from the object. In the last decades, methods for mapping chemical elements present across objects have been developed which not only give information about the materials present but also their location on the object\sidenote{For example, macro-\gls{XRF} scanning and hyperspectral visible spectrum imaging have been applied to Van Gogh's paintings (Chapter 1, section \ref{sec:ligth-sensitivity studies}).}. Information about the light sensitivity of materials (Figure \ref{fig:characterisation_functions} - b) can sometimes be found in the existing literature, or otherwise can be obtained with accelerated light ageing systems such as a light-ageing box or a \gls{MFT} system. Although light-ageing boxes can illuminate large areas if not the entire surface of a sample, applying the results to real objects is problematic. In the past decades, the emergence of "historically accurate" samples has reduced the gap between reconstruction samples and real objects, but even the best "historically accurate" sample cannot be compared with the materiality of a real object. On the other hand, micro-fadeometry allows for in-situ analyses avoiding the problems due to the use of reconstruction samples. Nevertheless, single spot analyses raise their own issues : how representative are analyses performed on tiny spots for the behaviour of the overall surface of the artwork? Even within uniformly perceived areas, there can be differences in thickness due to the brush movements of the artist, or slight variations in pigment concentration, which ultimately influence the local light-fastness properties of the paint. Moreover, inconsistencies in the formulation of commercial tube paints sold to artists in the late nineteenth century is another factor influencing colour change in their works.\\

\begin{figure*}
\centering
\includegraphics[width=\linewidth]{Chapters/Chapter1_General_Introduction/Figures/Functions_cc.png}
\caption[\hspace{0.3cm}Functions involved in the characterisation phase]{Functions involved in the characterisation phase. The curves have been added for illustration purposes; in practice, they differ from one application to another.}
\label{fig:characterisation_functions}
\end{figure*}

Based on the information provided by the characterisation, it becomes possible to outline a damage function that connects the exposure dose to the loss of values (Figure \ref{fig:characterisation_functions} - c). For example, Ella Hendriks and Agnes Brokerhof attempted to define the damage function taking Van Gogh’s painting of The Bedroom as a case-study (Inv. Nr. F0482, Van Gogh Museum). They found that a loss in value of 10\% over a period of 30 years (roughly corresponding to 4.8 \unit{\mega\lux\hour}\sidenote{With an illuminance of 50 lux and an exposure of 9 hours per day and 360 days per year.}) was considered an acceptable choice by the participants involved in the study \citep{hendriks_valuing_2017}.\\

There are two kinds of factors that influence damage on objects: external and internal factors. The first type is related to the environmental conditions in which the object is kept in terms of optical radiation (\acrshort{UV}-light-\acrshort{IR}), temperature (heat), relative humidity (moisture), oxygen concentration and pollutants. David Saunders provides an excellent summary of information on the damage of objects caused by each of the afore-mentioned factors \citep[Chapter 4, p.81-124]{saunders_museum_2020}. The damage due to optical radiation varies according to three parameters: the spectral power distribution, the intensity level and the exposure duration. It has been shown that both the presence of \gls{UV} radiation \citep{harrison_report_1953, mclaren_spectral_1956,padfield_control_1966} and the colour of the surface strongly influenced light-fastness properties \citep{saunders_light-induced_1994}. Similarly, the higher the intensity and the longer the exposure duration, the greater the colour change. Internal factors relating to damage mainly concern the material composition of objects, depending on which components are present and in which concentrations and how they affect the light sensitivity of the object. By cross-checking the data obtained from analysis of the object with information from the literature, our understanding of the contribution of each factor to our perception of damage improves.\\

On the ‘seeing’ side of the light dilemma (Figure \ref{fig:lighting_policy_framework}), the exhibition context requires the characterisation of three main elements: the object specifications, the lighting system and the observer. The object specifications, such as colours, glossiness, level of details, size, etc., can be determined by a careful examination of the object usually with the help of common tools such as a binocular microscope. Though there are many parameters that characterise light sources, only a few are used in the field of heritage conservation, such as the spectral power density, the illuminance and irradiance values and the \gls{CRI}, to name the most common ones. These parameters can be measured with various devices, such as spectrometers, luxmeters and radiometers\sidenote{A luxmeter gives photometric values (lumen or lux) while a radiometer expresses the output in terms of radiometric units (\unit{\watt\per\square\metre} or \unit{\joule\per\square\metre}). An actinometer measures the number of photons in a beam integrally or per unit time. This device is rarely used in the heritage field.}. The light sources can be characterised  on two scales - local or overall\sidenote{An overall scale assessment consists of the sum of several local measurements.} - in both cases measuring either at intervals, or continuously. The local scale corresponds to a single measurement performed at a precise location, used for example to assess the level of light falling on a known light-sensitive object in the context of an exhibition. Alternatively, overall scales of measurement entail assessing the distribution of light levels across entire exhibition spaces\sidenote{This approach won't be detailed in this thesis - however, it has been thoroughly reviewed and applied elsewhere (\citealp[38-41]{del_hoyo-melendez_study_2010}; \citealp{del_hoyo-melendez_evaluation_2011}).}. Recording information about visitors, such as gender, age, purpose of visit, etc., enables the museum to focus on the needs of a specific category of visitors. For example, if the museum notices that on Mondays most visitors are elderly people, the museum could decide to increase the intensity of the lighting system by 20\% every first Monday of the month, so that visitors could properly see the objects and ultimately gain better access to their values. \label{par: light characterisation} \\



Understanding the 'seeing' side of the balance requires  knowing which parameters influence the visibility of an object and how. Visibility is directly linked to the visual performance of the observer\sidenote{Visual performance is usually defined as the speed and accuracy of visual processing, for which a metric called \gls{RVP} has been developed \citep{rea_relative_1991}).}, but one might ask; which parameters influence the visual performance of the observer and in what way? Since we still lack a comprehensive understanding of the influence of the parameters at play, presently we can only partly answer this question. \citet{saunders_museum_2020} provides a concise and clear overview of the main factors of influence on human visual performance\sidenote{See \citet[section 1.6 (p.33-38) and chapter 6 (p.156-176)]{saunders_museum_2020}. Chapter 9 in \citet{saunders_museum_2020} gives practical recommendations for cultural institutions in order to maximise the visibility.}, namely:  the age of the observer, the intensity of the light source, the level of detail, and the level of contrast of the object. He further points out that visual performance decreases with raised level of detail or age, while it increases as illuminance values and contrast increase. \\


\subsubsection{Predict}
\label{sec:predict}

The next section reviews two tools used for predicting colour change: digital visualisations and preservation targets. \\

The first colour change prediction image was enabled by using one of the first micro-fading devices that had been developed a few years earlier. In their article, \citet{morris_virtual_2007} virtually faded several paintings based on microfading measurements. Although their methodology and the technique presented some limitations \citep[224]{morris_virtual_2007}, their results clearly show our technical ability to achieve this. With the recent development of imaging techniques, some of the limitations they faced, especially concerning the representation of a spot measurement to the entire surface of a similar colour, could now surely be partially overcome. More recently, in her article on preservation targets, Pesme laid the theory that connects virtual faded images to the context of light management. Again, based on the results of microfading analyses, she virtually faded a digital image in order to illustrate the change of colour upon light exposure \citep[Figure 4]{pesme_presentation_2016}. Also in 2016, Ella Hendriks and Agnes Brokerhof organised the first of several workshops where the first virtual colour change prediction of a work by Van Gogh – The Bedroom (Inv. Nr. F0482, Van Gogh Museum) – was submitted to the judgement of various participants \citep[3, Figure 3]{hendriks_valuing_2017}. Compared to the earlier (2010 and 2013) visualisations of the former colours of the Bedroom, a less elaborate method was used to create the  virtual predictions of the painting’s future colours. Notwithstanding this simplification,  this first attempt on a van Gogh painting already facilitated a dialogue about the exhibition and preservation of the Van Gogh Museum collection over the long term, enabling the authors to define a preservation target for the collection. \\

A \gls{PT} is defined as a ‘minimum duration of stewardship during which a \gls{JND} in colour is considered acceptable to happen on the item surface’ \citep{pesme_setting_2016}. The notion of acceptability in the context of colour change is difficult to define. Two studies correlate the fact that the colour change acceptance is mainly dependent, among many other factors, on the colour and size of the faded surface. For example, for comparable amounts of changes in the blue and the red, most observers accept a longer time period for the red (\citealp[86]{brokerhof_optimum_2008}; \citealp{richardson_acceptable_2007}). In addition, the location of colour changes on the object also influences its appreciation: in a portrait, for instance, fading is found more disturbing in the flesh tones compared to the background area \citep{richardson_acceptable_2007}. The notion of just noticeable difference corresponds to the threshold above which a normal observer is able to distinguish the difference between two coloured patches viewed side-by-side under good lighting conditions (\citealp{crawford_just_1973}; \citealp[note 6]{pesme_presentation_2016}). In theory, one \gls{JND} is approximately equal to 1.6–1.8 \dEOO  (\citealp{michalski_light_2018}; \citealp[21]{cie_technical_committee_3-22_control_2004}) but in practice, a colour difference can already be observed with a \dEOO below one in many cases\sidenote{For example, I was able to observe a difference in colour on chrome yellow samples for which a \dEOO of 0.8 had been reached when micro-fading the samples.}. Characterising the preservation target involves the use of digital visualisations and is achieved by constant communication between all actors related to the artworks, ranging from the curatorial department to museum visitors. As an example, a workshop organised at the Van Gogh Museum in 2017 came to the value of one \gls{JND} in 30 years when using \textit{The Bedroom} (Inv. Nr. F0482, Van Gogh Museum) as a case study \citep{hendriks_valuing_2017}. \\

\begin{figure*}[!h]
\centering
\includegraphics[width=0.8\textwidth]{Chapters/Chapter1_General_Introduction/Figures/PT_exp-dose.png}
\caption[\hspace{0.3cm}Relating the fading data to the preservation target]{Relating the fading data to the preservation target.}
\label{fig:PT}
\end{figure*}

Combining the fading data with the \gls{PT} gives us a light budget, with a time constraint provided by the \gls{PT}, and which corresponds to the amount of exposure dose after which one \gls{JND} will occur (Figure \ref{fig:PT}). Obtaining the fading data can be achieved by two methods: literature comparison or artificial light ageing analysis, \ie microfadeometry. Identification of the materials enables the researcher to find information about their respective light sensitivity. Defining the energy value with microfadeometry can be done directly if the energy of the fading beam has been characterised with a photometer/radiometer or indirectly by comparison with reference fading materials, \ie blue wool standards, as performed by \citet[803–4]{fieberg_paintings_2017}. \\



\subsubsection{Prevent}

The only way to fully prevent light-induced colour changes is to store the objects in complete darkness. As such drastic measures are not always possible, the prevention step contains a set of measures in order to control and optimise the rate of colour change on objects – a list of measures can be found in \citet[279, Table 10.3]{saunders_museum_2020}. Some of the measures such as the elimination of \gls{UV} and \gls{IR} radiation are fairly straightforward, their negative effects on objects having been known for a long time by conservators and scientists \citep{feller_controeffets_1964,feller_speeding_1975, padfield_control_1966} while their removal does not impact the visibility of the viewer. In contrast, the choice of lighting parameters, such as the illuminance values and the duration of exposure is more challenging. Starting from the preservation target mentioned above, the institution can manage the exposure of its collections within the limit set by the light budget. The museum can either choose a fixed illuminance value and then calculate the recommended exposure duration, or add some flexibility according to the situation. As Michalski stated, referring to Thomson, the system requires some flexibility in order to render it efficient in practice \citep[97]{michalski_lighting_1997}. When a lighting policy is mainly focused on illumination levels in order to control the colour changes, it lacks flexibility. For example, it is often stated that highly light-sensitive objects should never be exposed to intensity levels above 50 lux. But exposing such objects to 50 lux for 3 months is no better or worse than exposing them to 150 lux for 1 month; in fact, it might even be preferable if it allows persons with low vision abilities to see the objects more clearly. Referring to an illuminance level on its own does not mean much. Instead, we need to think in terms of exposure dose (\unit{\lux\hour} or \unit{\joule\per\square\metre}) and give a similar level of importance to both the vulnerability and the visibility parameters in our lighting policy, as illustrated in the table provided by \citet[Table 1]{michalski_lighting_1997} and the chapter on this issue by \citet[Chapter 10]{saunders_museum_2020}. \\

Although choosing the lighting parameters in the framework of a risk-based methodology gives more flexibility, it also requires a closer monitoring of the illumination history over a long period of time. While the idea seemed unachievable a few years ago, I believe that in the following decades it could become possible, at least for individual objects. Exhibition spaces usually tend to be enclosed spaces illuminated by \gls{LED} lamps, meaning that the illumination levels can be monitored more closely. For example, during the exhibition \textit{Van Gogh and the Sunflowers}\sidenote{Held at the Van Gogh Museum 21 June–1 September 2019.}, the total exposure dose recorded was 37 600 \unit{\lux\hour}\sidenote{Kees van den Meiracker, former Head of Conservation at the Van Gogh Museum, personal communication, 24/10/2019.}. For the exhibition, the Van Gogh Museum commissioned an artist and trained conservator, Charlotte Caspers, to make a partial reconstruction of the top right quadrant of the \textit{Sunflowers} painting (Inv. Nr. F0458, Van Gogh Museum) using historically accurate paints for part of the reconstructed area. Since Caspers created the reconstruction in May/June 2019, the Van Gogh Museum has kept track of its illumination history and regularly measures the colours at a series of agreed spots on the painting. While it is feasible to monitor a single object, it does not seem possible (at least for now) and perhaps not even relevant\sidenote{Assessment of the costs and benefits of monitoring, as well as the importance of the objects in question should be conducted prior to establishing monitoring systems.} to monitor entire collections. However, a museum could easily select a smaller group of high value objects in the collection and monitor their illumination history\sidenote{The Van Gogh Museum currently collaborates with ASML on the issue of environmental monitoring of the on-display collection (ASML-Van Gogh Museum Partnership in Science).}. \\

The most accurate way to monitor light levels is to use an electronic device, as was mentioned on page \pageref{par: light characterisation}. Another possibility relies on the use of blue wool standards (ISO Standard R 105; British Standard BS 1006), currently the only dosimeter system available in the field of cultural heritage\sidenote{A series of promising light dosimeters developed and implemented in the past \citep{bacci_lightcheck_2005, dupont_development_2008, lavedrine_blue_1998}, are, unfortunately, no longer available.}. They have been used from the start in microfadeometry in order to rank the lightfastness of materials\sidenote{For more information on the use of blue wool standards in microfadeometry see Chapter 2, section \ref{sec:dosimeter}.}. Several aspects of the \gls{BWS} have been studied since they were first introduced to the field of cultural heritage in the 1970s by Robert L. Feller (1919–2018) \citep{feller_use_1978,feller_further_1978, feller_continued_1981}. Publications report on their practical use in cultural institutions \citep{bullock_measurement_1999, tennent_light_1987}, on studies on their wavelength sensitivity \citep{hattori_wavelength_2012} and on their behaviour towards the reciprocity principle \citep{del_hoyo-melendez_investigation_2011}. \\


\newpage
\section{Light-induced colour changes in Van Gogh’s works}

\subsection{An overview of past studies}
\label{sec:overview_past_studies}

This section gives the reader an overview of past studies on colour changes in Van Gogh's works\sidenote{A list of all published research projects on colour changes in Van Gogh's works as well as a chronological chart are given in Appendices \ref{app:ch1_timeline_VGM-projects}.}, so that this PhD research can be connected to previous research projects. The section has been structured in two main parts according to the materiality of the objects: paintings on the one hand and paper-based objects on the other. The types of studies conducted can themselves be divided into two groups that will each be reviewed for the paintings and paper-based objects. The first group of studies characterises the light sensitivity of the materials present in the objects (colourants, binding medium, support, etc.) followed by digital visualisation projects that mostly aim to reconstruct the original appearance of faded colours. Due to their complementary aspect, many research projects include both types of study. However, before turning to these examples, it is essential to address three overarching aspects that will help to understand and connect these past research projects to one another. The first aspect considers the broader historical framework in which these studies took place, the second deals with their justification, while the last aspect broadly addresses the research questions that were developed during these projects. \\

From a chronological perspective, all the research projects on colour changes in Van Gogh's works – except some chemical analyses performed around 1967 in the Laboratoire des musées de France \citep[17, note 7]{cadorin_colour_1991} – occurred from the mid-1980s onward. At first glance, this may seem surprising, as knowing that Van Gogh was aware of the light sensitivity of some of the paints he used\sidenote{Letters 595 and 765 \url{http://vangoghletters.org/vg/} (accessed 19/12/2020).}, we might expect this issue to have been investigated much earlier. Regardless of this insight, in fact the risk of colour changes in his works has not, and could not really have been considered a major issue for the owners and caretakers of the collection until the late 1990s. One reason was that up until the 1970s there was a lack of scientific knowledge and expertise on the topic among cultural heritage professionals, so that only simple protective measure could be taken\sidenote{The most common measures consisted of reducing the light intensity by lowering roller blinds in front of windows \citep[141]{brommelle_russell_1964} and applying \gls{UV} blocking filters \citep[91–2]{feller_controeffets_1964}.} While awareness that light can induce damage to materials used for artworks has been known since Antiquity\sidenote{In his book on architecture, Vitruvius (1\textsuperscript{st} century BC) mentioned the light-induced darkening of vermilion \citep[Chapters VIII and IX]{vitruvius_vitruvius_1914}. The book is available on the website: \url{http://www.perseus.tufts.edu/hopper/}, accessed 25/08/2023.}, the first major scientific report on the action of light on artefacts in museum collections was only written by Russell and Abney in 1888 \citep{brommelle_russell_1964}\sidenote{Previous experiments on similar topic by Dufay in 1733 and Field in 1808 are mentioned in the literature by Padfield and Harley respectively \citep{harley_fields_1979, padfield_light_1966} but unfortunately original detailed sources of these experiments have not yet been found so we do not know exactly what kind of experiments and results both Dufay and Field obtained.}. Additionally, most experiments on light ageing were conducted after 1950\sidenote{For a good review on light fading studies, see \citet[14–26]{del_hoyo-melendez_study_2010}.}, while the first academic training programmes for conservation only emerged in the 1970s, equipping conservators with the expertise to link outcomes from scientific experiments to conservation issues\sidenote{For example, in France the first educational programme for conservators was created in 1973 by the Panthéon-Sorbonne University (\url{https://www.icosaedreparis1.com/copie-de-qui-sommes-nous} (accessed 10/02/2023). It should however be noted that the involvement of science in conservation and collaboration between conservators and scientists took off much earlier, becoming wide scale especially in the period 1930-1950.  See, for instance \citet{dupre_histories_2022}.}. Lastly, while a permanent environment for housing the Van Gogh family collection was realised with the construction of the Van Gogh Museum that first opened its doors to the public in 1973, it would take more than a decade before a professional care of the collection was established in keeping with modern standards of conservation. A turning point was the first condition survey of the paintings collection performed in the mid-1980s, which can be considered as the starting point of systematic observation and monitoring of the collection. The outcome of the survey was used to advocate for the need for a permanent in-house conservator, leading to the appointment of Cornelia Peres in 1986 \citep[35–6]{hendriks_vincent_2011}. Tangible evidence of colour change was gathered in the decades that followed, often observed when unframing objects and comparing areas of colour covered by the frame or passe-partouts around the edges with parts exposed to light. For instance, Figure \ref{fig:VG_works_cc} shows examples of colour change in paintings and drawings documented during the \gls{REVIGO} project conducted in 2013–2017. Another example is the traces of colour preserved on the edge of Van Gogh’s \textit{The Garden of the Asylum} (November 1889, F0659) noted during an extensive technical and stylistic examination of the painting that took place in 1999 due to questions about its attribution \citep[Figures 26 and 27]{hendriks_van_2001}. \\


\begin{figure*}[!h]
\centering
\includegraphics[width=\linewidth]{Chapters/Chapter1_General_Introduction/Figures/VanGogh_works_colour-changes_lowres.png}
\caption[\hspace{0.3cm}Examples of colour changes on Van Gogh's works]{Examples of colour changes on Van Gogh's works: F0409(a), F1423(b) (\copyright Van Gogh Museum, Amsterdam (Vincent van Gogh Foundation)).}
\label{fig:VG_works_cc}
\end{figure*}

Over the past 30 years, the Van Gogh Museum has invested considerable time and energy on colour change research projects\sidenote{A timeline of the research projects carried out at the Van Gogh Museum is available in Appendix \ref{app:ch1_timeline_VGM-projects}.}. The justification of such commitment seems a legitimate question to ask: why is the topic of colour change that important to the Van Gogh Museum? It is first essential to realise that the Van Gogh Museum is a museum centred around the life and work of Vincent van Gogh, in other words the museum makes the connection between the visitor, the artist and his work, as implied by the mission statement of the  museum\sidenote{‘The Van Gogh Museum inspires a diverse audience with the life and work of Vincent van Gogh and his time’ (\url{https://www.vangoghmuseum.nl/en/about/organisation/mission-and-strategy;} (accessed 06/01/2023).}: it acts as an open door to the past, allowing us to ‘meet’ Vincent van Gogh and experience the historical and artistic context in which he lived. In other words, through the lens of the Van Gogh Museum, visitors are provided with knowledge of Van Gogh and his time that they can potentially use in their own lives (Figure \ref{fig:VG_observer}). Although knowledge of past events is never complete and there is always an element of interpretation, the museum aims to present a historical version that is as close to reality as possible, but is automatically confronted with the need to address notions such as truth, authenticity and original intent, which have been subject to extensive discussion \citep{munoz_vinas_ethics_2020,leveau_problemes_2009,price_historical_1996}.\\


\begin{figure*}
\centering
\includegraphics[width=0.6\linewidth]{Chapters/Chapter1_General_Introduction/Figures/VGM_lenses.png}
\caption[\hspace{0.3cm}Relationship observer - Van Gogh]{Relationship between observer and Van Gogh through the lens of the Van Gogh Museum.}
\label{fig:VG_observer}
\end{figure*}

To what extent do colour changes on Van Gogh’s works affect the mission of the museum and more precisely, how do colour changes relate to the aforementioned notions when applied in the context of Van Gogh’s works? If ‘truth’ is defined here as the original appearance of an object, then any changes in surface colour increase the $\alpha$ value on Figure \ref{fig:VG_observer} and drive us further away from the ‘truth’. Consequently, if we expect the colour of Van Gogh’s works to appear as they were originally, it does make the works appear ‘less authentic’. However, as \citet[24-7]{munoz_vinas_ethics_2020} noted, authenticity is strongly dependent on the expectations of the viewer. When colour change is perceived to diminish the authentic appearance, it does not necessarily make other aspects of an object less authentic; a faded painting created by Van Gogh is still considered as an ‘authentic’ Van Gogh\sidenote{In my opinion, it does not make sense to simply qualify objects as ‘authentic’. An object can be defined as an ‘authentic oil painting’, if it turns out that it is indeed an oil painting. The notion of authenticity should always refer to specific aspects or properties of an object and establish a correspondence between a given statement and the reality of what it is \citep[23]{munoz_vinas_ethics_2020}.}. Ernst van de Wetering reviewed the notion of the artist’s intent in the context of conservation activities on Van Gogh’s paintings, from which he concluded that ‘restoration has a certain autonomy independent, to some extent, from the artist’s intentions’ \citep[196]{van_de_wetering_autonomy_1996}\sidenote{This essay was first published in 1989 by \citet{van_de_wetering_autonomy_1989}.}. Although the Van Gogh Museum wishes to stay as close as possible to the appearance of the paintings when they left Vincent’s easel, it will never be possible to return to the original appearance of the paintings, since colour changes are irreversible phenomena. Nonetheless, digital visualisation tools allow us to estimate the original colours of faded objects virtually. However, yellowed varnish (tinted or not) on Van Gogh's French period paintings are commonly removed because they are considered as 20\textsuperscript{th} century taste rather than the true intention of Van Gogh, whose usual preference was to leave his works of the period unvarnished. Therefore their removal is considered to be an acceptable practice that brings us closer to the original appearance of the works of art provided that it can be safely accomplished without damaging the original paint\sidenote{Ella Hendriks, personal communication, 02/08/2020.}. In-depth study of sources including Van Gogh’s own correspondence provides a global understanding of the artist’s original intentions regarding the use of colour in his works \citep{dijk_van_2013, heugten_working_2003}. While Van Gogh’s use of colour evolved throughout his career, reflecting both his experimental mindset and his advancing skills as an artist, some constant elements provide coherence to his artistic practice seen as a whole. First of all, colour was an essential expressive medium for Van Gogh, mainly inspired by the example of Delacroix's paintings and his numerous readings on the subject\sidenote{The readings of Van Gogh included the following books: Armand Cassagne, Traité d'aquarelle \citeyearpar{cassagne_traite_1875}; Charles Blanc, Grammaire des arts du dessin: architecture, sculpture, peinture \citeyearpar{blanc_grammaire_1883}; Félix Bracquemond, Du dessin et de la couleur \citeyearpar{bracquemond_du_1885}; \citep{dijk_van_2013, heugten_working_2003}.}. Secondly, his bright use of colour was essentially based on the theory of simultaneous colour contrast, where the juxtaposition of a primary colour with a secondary colour made by mixing the two other primary colours intensifies the contrast\sidenote{Theory developed in the 19\textsuperscript{th} century by Michel Eugène Chevreul (1786–1889).}. The application of this theory, which he often mentioned in his letters\sidenote{See letters 494, 536, 622, 634 (\url{http://vangoghletters.org/vg/;}, accessed 18/01/2023).}, can be appreciated especially in his paintings of the French period, although today the effect has often become less pronounced. Applying steps 3–5 for determining a discrepancy between the physical condition and the meaning of an object in the SBMK decision-making model \citep[8–14]{giebeler_decision-making_2019} to consider colour change in Van Gogh's works enables us to target a main question: how does the meaning of Van Gogh's works and the representation of his artistic intent change as a result of colour change? In the case of simultaneous colour contrast, when at least one of the two juxtaposed colours decreases in intensity, this immediately affects the contrast between them. As a consequence, the original intention of vibrant colours desired by Van Gogh can no longer be appreciated, at least not to its full extent, which inherently impacts the mood that he wanted to convey through his paintings. Partial loss of meaning of an object due to colour change deprives the observer of the full ‘Van Gogh experience’, which ultimately explains why the museum invests in research projects aimed to understand and control colour change phenomena in the works in its care.\\



\subsubsection{Light-sensitivity studies}
\label{sec:ligth-sensitivity studies}


All studies on light sensitivity aim to answer at least one of the questions on the following list (not exhaustive). Most of these questions have been partially answered, providing us with a fair understanding of the phenomena involved.\\

\begin{itemize}
    \item What? Are there any light-sensitive materials present in the object? If so, which materials are considered to be light-sensitive? Are there any ingredients present in the object that might foster, or slow down the rate of colour change?
    \item Where? Where are these light-sensitive ingredients located on the surface of the object?
    \item How? How light-sensitive are they and can we characterise the kinetic properties of their colour change behaviour? Which parameters influence their colour change behaviour? How do these changes impact the overall appearance of the object? 	
\end{itemize}

\vspace{0.5cm}

\textit{\underline{Paintings}}


Most light-sensitivity studies on paintings have investigated the fading/darkening properties of red and yellow paint layers as both contain some of the most fugitive classes of red lake\sidenote{Besides red lakes, red lead and vermilion have also been investigated. Both pigments have been found in paintings by Van Gogh \citep{bommel_investigation_2005, hendriks_van_2019} and used in light-sensitivity experiments \citep{monico_chemical_2019}. For more information on these pigments see \citet{feller_studies_1967, gettens_vermilion_1993, west_fitzhugh_read_1986}.} and chrome yellow pigments. Other factors, such as the type of ground layer or binding medium and the impact of past conservation treatments, may also influence the colour appearance of an object. One example that will be returned to later in this section is the study by \citet{nieder_colour_2011} on colour change in canvas and ground layers as a consequence of wax-resin lining and varnishing conservation treatments.\\

Past research projects share three, often combined, complementary methods used to investigate the light sensitivity of Van Gogh's works: 

\begin{itemize}
    \item Visual comparison
    \item Chemical analysis
    \item Light-fading experiments
\end{itemize}


The most straightforward way to assess colour shifts over time consists of a visual comparison of the painting at intervals, starting from the moment of its creation up to the present day, with the help of photographs. One example of this approach is the research that took place for the exhibition, \textit{Van Gogh: Irises and Roses}, held at the Metropolitan Museum in New York in 2015\sidenote{MET Inv. Nr. 58.187 and 1993.400.5. More information is available on the following websites: \url{https://www.metmuseum.org/exhibitions/listings/2015/van-gogh} and \url{https://www.youtube.com/watch?v=wJM3JhAfad8} (accessed 28/06/2023).}. Combining such macroscale information with microscale analyses can improve our understanding of colour change phenomena by creating a bridge between observation and chemical processes. However, while photographs may clearly reveal loss of colour, it is important to bear in mind that the observed colour differences may be influenced by the specific photographic technique employed and by ageing of the photographs themselves. Furthermore, quantitative assessment of the extent of colour change using photographs can be quite ambiguous, limiting their usefulness for interpretation. Nevertheless, the approach can give us a good general sense of the phenomenon.\\

The second approach relies on chemical analyses and light microscopy observations of cross-section samples in order to identify the materials present in the object. This is often combined with information on the light sensitivity of those materials derived from the literature or from light-ageing experiments. Most research projects have exploited the new range of scientific tools that became available in the cultural heritage field in the course of the 20\textsuperscript{th} and 21\textsuperscript{st} centuries. While full consideration of each application of these tools in the context of Van Gogh's works lies beyond the scope of this thesis, this paragraph offers a brief summary of the main developments and outcomes\sidenote{A brief summary of the main research projects is given as a table in Appendix \ref{app:ch1_timeline_VGM-projects}.}. The first thing to notice is the multitude of analytical techniques that are available, offering different means to identify the composition of materials; for example, \citet[271]{geldof_van_2013-3} mentions five techniques used to identify eosin in a painting. The range is needed since a single technique cannot be used to characterise all the materials present, instead researchers must often combine several methods to identify the full palette used by a painter. For instance, while \gls{XRF} can detect heavy elements such as zinc, lead, or iron, \gls{HPLC} is needed to reveal the presence of organic dyes. A second trend to note is that scientific tools have become more precise and powerful over the years. From the 1980s onwards, scientific techniques were used qualitatively in order to identify the presence of specific elements in Van Gogh’s paintings. For example, \citet{cadorin_decoloration_1987} found geranium lake (eosin) in a painting by detecting the presence of bromine with \gls{SEM-EDX} analyses. With the same technique Rioux identified eosin and cochineal lake in several paintings from the collection of Doctor Gachet \citep[406]{rioux_caracterisation_1999}. \gls{HPLC} coupled with \gls{PDA} detection was also mainly used during the ‘Red Lakes’ project (2000–2005) in order to distinguish the various red lakes \citep{bommel_investigation_2005}. In addition, new information on the paint composition has been revealed with the implementation of other analytical techniques. For instance, while \gls{XRF} and \gls{SEM-EDX} were used extensively during the ‘Studio Practice’ project in order to identify pigments and determine the substrate of red lakes \citep{geldof_van_2013-3, geldof_van_2013-1, geldof_van_2013}, \gls{GC-MS} enabled the identification of the binding medium \citet[277-8]{geldof_van_2013-3}. An estimation of the composition ratio was also performed on a geranium lake paint tube from Tasset et L'H\^{o}te with \gls{PIXE} and \gls{RBS} \citet[284-5]{geldof_van_2013-3}. Finally, recent tools allowed us to obtain elemental mapping of an artwork by using \gls{MA-XRF} and \gls{MA-XRPD} scanners. For instance, several research projects were able to map the presence of specific elements characteristic of certain pigments used in Van Gogh’s paintings (\citealp[Figures 1, 4 and 5]{vanmeert_chemical_2018,centeno_van_2017}; \citealp[107, Figure 4.11]{hendriks_methods_2019}). In conclusion, the chemical techniques now available allow us to characterise qualitatively, quantitatively and spatially the materials present in an object and enable us to identify the pigment palette used by Van Gogh. Such information provides a basis that can feed into the making of complex digital visualisations, as the next section will explain.\\

The last investigative method dealing with the light sensitivity of Van Gogh's works entails accelerated light-ageing experiments. It was first used during the Red Lakes project\sidenote{A more detailed description of this project is available in Appendix \ref{app:ch1_Red-lake_project}.}, which, as the name suggests, focused on the colour change behaviour of red lake samples. A main outcome of the study was to show that in a worst-case scenario, a complete loss of colour of an eosin-based lake could occur with an exposure dose equivalent to ten years of ageing under museum conditions \citep{burnstock_comparison_2005,van_den_berg_fading_2006}\sidenote{With an illuminance value of approximately 150 lux during 3600 hours per year.}. Such fugitive colours on Van Gogh's paintings likely faded away soon after the death of the artist, if not before. While during the Red Lakes project model paint-outs were faded with fluorescent lamps inside a light ageing box, \citet{fieberg_paintings_2017} is the only publication to date that used microfadeometry to study the light-ageing properties of Van Gogh painting materials. It is also the first and only published study to have assessed the light sensitivity on an original, aged paint layer. The single microfading analysis was performed on an eosin-based paint in a cross-section sample taken from the painting \textit{Undergrowth with Two Figures}\sidenote{The painting belongs to the collection of the Cleveland Museum of Art (USA) and is assigned the number F773 in Van Gogh’s raisonné \citep{de_la_faille_works_1970}.}. The substrate of this geranium lake sample seems to be lead(II) sulphate, which is quite rare as aluminium is usually detected in Van Gogh's eosin paint layers \citep{geldof_van_2013-3}. The results assigned a \gls{BWSE} of 2.5 resulting in one \gls{JND} after a 6-year period of illumination at 150 lux for 8 hours per day \citep[804]{fieberg_paintings_2017}. This is equivalent to one \gls{JND} after a 14-year period of illumination at 50 lux for 10 hours per day, which is half the time estimated by \citet{hendriks_valuing_2017}. In other words, if the museum sets a preservation target of one \gls{JND} every 30 years, the results of \citet{fieberg_paintings_2017} indicate that with the current lighting conditions, either the intensity of the spotlights or the duration of exposure would have to be reduced by half. However, these are the results of only one analysis on one of the most light-sensitive pigments. A better assessment of the risk of light-induced colour change requires data on all pigments. In other words, further microfading analyses, performed on original Van Gogh works, are needed to fill the information gap so that the museum can properly assess the risk of light-induced colour change.\\

Regarding the yellow pigments used by Van Gogh, past studies have mainly concerned the family of chrome yellow pigments. Although the instability of chrome yellows towards light exposure has been reported since early in the 19\textsuperscript{th} century\sidenote{For historical overviews of studies on chrome yellow degradation see \citet[20–21]{monico_degradation_2012} and \citet[87–90]{otero_historically_2018}.}, a good understanding of their fabrication processes and degradation mechanisms was only gained in the two first decades of the 21\textsuperscript{st} century through research led by Vanessa Otero at the NOVA University of Lisbon and Letizia Monico at the University of Perugia. They demonstrated that the darkening of chrome yellow paint layers was due to a reduction of the original Cr(VI) to Cr(III) ion \citep{monico_degradation_2011,monico_degradation_2011-1}. In addition, the work of Vanessa Otero revealed the influence of the manufacturing conditions, especially the pH of production of PbCr\textsubscript{1-x}S\textsubscript{x}O\textsubscript{4}, with 0.4 $\leq x \leq$ 0.5, on the photostability of the final paint product \citep[Chapter 4]{otero_historically_2018}. Her studies also demonstrated the importance of ‘historically accurate’ reconstructions for facilitating the interpretation of light-ageing results in connection with real artworks \citep{otero_nineteenth_2017}. Analysis of Van Gogh's works has confirmed the presence of three chrome yellows, designated as types 1, 2 and 3 in Van Gogh's paint orders and letters\sidenote{As a few examples, see the letters n$^\circ$0593, n$^\circ$595, n$^\circ$688 and n$^\circ$740 (\url{http://vangoghletters.org/vg/;} accessed 16/11/2020).} which corresponds to the current nomenclature of ‘lemon’, ‘middle yellow’ and ‘orange’ shades respectively \citep{monico_degradation_2013}. A relative assessment of their light sensitivity showed that the lemon chrome yellow (type 1) is prone to exhibit greater colour change upon light exposure compared to the two other types \citep{monico_degradation_2013-1}. The influence of different light sources – such as \gls{LED}, halogen and xenon – on the colour change behaviour of chrome yellow samples has also been tested, highlighting the risk of halogen and xenon light sources compared to warm white LEDs \citep[149–54]{monico_chemical_2019}. The research on chrome yellow used in \textit{The Sunflowers} painting (F0458) formed one part of a comprehensive campaign of examination and condition assessment that took place within the so-called ‘Sunflowers’ project led by Ella Hendriks from 2012 to 2019 \citep{hendriks_van_2019-1}. Two presentations took place as a result of this research, the first being a focused display of \textit{The Sunflowers} held at the National Gallery in London in 2016 \citep{ashok_van_2016} and the second an exhibition \textit{Van Gogh and the Sunflowers} held at the Van Gogh Museum in 2019. The opening of the exhibition was marked by an international symposium and the launch of the first issue of the Van Gogh Museum Studies series which focused on \textit{The Sunflowers} and provides a compilation of the research and findings from the last three decades of study \citep{hendriks_van_2019}. In both \textit{The Sunflowers} exhibition and accompanying publications, the topic of colour change and its implications for the condition and display of the paintings has been given particular attention. \\

Aside from pigments, using historically informed reconstructions, \citet{nieder_colour_2011} conducted research on colour change in canvas and ground layers as a consequence of wax-resin lining and varnishing conservation treatment of Van Gogh’s paintings. This study illustrates the fact that other factors besides light exposure can induce colour changes, and underlines the impact of human intervention on the visual appearance of artworks. In addition, the researchers highlighted other factors that have a major impact on the colour change phenomenon, including surface grime, natural discolouration of the oil binder in the paint, and the type of binding medium used for the ground. \\


\textit{\underline{Paper-based objects}}

\vspace{0.5cm}

\begin{figure*}[!h]
\centering
\includegraphics[width=0.8\linewidth]{Chapters/Chapter1_General_Introduction/Figures/stratigraphy_paper.png}
\caption[\hspace{0.3cm}Schematic stratigraphy of paper-based objects]{Schematic stratigraphy of paper-based objects.}
\label{fig:paper_stratigraphy}
\end{figure*}

Paper-based objects include the letters of Van Gogh as well as his drawings; the latter can be made up of a loose sheet of paper or bound together in the form of a notebook or book. The structure usually consists of two elements: the support (paper, board, etc.) and the media (ink, graphite, etc.). According to the method of application (liquid or solid-based media) and the characteristics of the support (fibre type and size, porosity, sizing agent, etc.), the media can be mostly located on top of the support or spread inside the fibres of the paper (Figure \ref{fig:paper_stratigraphy}), which ultimately has an influence on the overall light sensitivity of the object. Due to the fact that fading is a surface phenomenon, as shown by \citet{johnston-feller_kinetics_1984}, when the media is mostly located above the support (Figure \ref{fig:paper_stratigraphy} - c), the latter has relatively little influence on the fading behaviour of the media, in contrast to a situation where the media is intermingled with the fibres of the support (Figure \ref{fig:paper_stratigraphy} - a). In such cases, the fibres can either act as a protective barrier or foster the degradation\sidenote{A similar reasoning could be applied to paints, where the top layers protect the layers below.} \citep{launer_photochemical_1943, lee_damaging_1988, lee_influence_1986}. In addition, the support of paper-based objects is often visible and therefore exposed to external environmental agents. In our case, it means that it is not possible to simply assess the light sensitivity of the media in order to estimate the behaviour of the overall object; the light sensitivity of the support should also be taken into account.



Although the photochemistry of paper has been the subject of much research\sidenote{Reviews of past studies have been published by \citet{havermans_photo_1997} and \citet{padfield_deterioration_1969}.}, the light sensitivity of Van Gogh's paper – whether of his letters or drawings – has never been investigated. However, a characterisation of the papers and media used by Van Gogh has been made \citep{ives_vincent_2005, vellekoop_vincent_2007}, which is a necessary preliminary step before any further investigation on light sensitivity can be carried out. Regarding the inks, the research projects of \citet{neevel_non-invasive_2008} and \citet{neevel_identification_2013} characterised the type of inks present in the drawings and letters of different periods. The results reveal that although Van Gogh generally used similar inks for his drawings and letters, there is shift of ink between his Dutch and French periods with a predominant use of iron gall inks at first followed by a more frequent use of logwood inks during his stay in France \citep[Figures 10 and 11]{neevel_identification_2013}. While the chemical analyses enabled the identification of various inks, they also suggest a certain complexity that our current knowledge on early synthesis processes of synthetic inks is barely able to grasp. Indeed, the manufacture of synthetic inks was still in its infancy at that time; the raw materials were not always pure and the recipes could vary greatly from one batch to another, even within the same company\sidenote{Frank Ligterink, personal communication, 29/04/2020.}.\\

Regarding the light sensitivity of these inks, there is abundant literature on iron gall inks in contrast with other types of inks used in the 19\textsuperscript{th} century. Although studies on the behaviour of iron gall inks were not performed within the framework of Van Gogh's artistic process, it still gives us a good estimation of how iron gall inks on Van Gogh's letters behave upon light exposure. It has been demonstrated that iron gall inks applied on paper are light sensitive, as they tend to become lighter and yellower \citep[181-2]{reissland_light_2002}. Later studies, in which microfadeometry was used to assess the light sensitivity, confirmed the results obtained by Reissland and Cowan \citep{tse_microfade_2010} as well as the observation of a post-illumination colour reversion first mentioned by Han Neevel \citep{ford_accelerated_2014}.\\

In the case of synthetic inks, the study by \citet{confortin_study_2010} aimed to understand the degradation process of crystal violet ink, which is often found in letters and drawings of Van Gogh. Similar degradation products of the ink were observed among different types of paper while outlining differences of fading rates from one type of paper to another. Moreover, the importance of gum arabic in the lightfastness of the ink was also denoted, suggesting a need for further research. More recently, a group of researchers within the TooCold project\sidenote{More about the TooCold project can be found on the following website: \href{https://www.uva.nl/en/shared-content/faculteiten/en/faculteit-der-natuurwetenschappen-wiskunde-en-informatica/news/2017/11/combatting-light-induced-decay-of-art-food-and-more.html}{TooCold project} (accessed on 17/06/2023).} developed a novel method - based on a \gls{LCW} cell - to study the chemistry of light-induced degradation of coloured materials \citep{den_uijl_comparing_2022}. Application of this new technique on eosin and crystal violet inks showed comparable results with samples aged in standard light-ageing protocols (Xenotest and SprectoLinker). While opening new possibilities for the study of light-induced colour change phenomena, this new system could be a relevant asset in order to help researchers relate chemical processes to physical changes visible on objects.\\


    
\subsubsection{Digital visualisation projects}

\textit{\underline{Paintings}}

The identification of light-sensitive pigments and their susceptibility to colour change upon light exposure, laid out in the preceding section, can serve as the basis for making digital visualisations of faded paintings. Different terms have been used to describe such images, including digital ‘visualisations’, ‘reconstructions’ or ‘rejuvenations’. The terms ‘reconstruction’ and ‘rejuvenation’ imply a technique aiming to recover a younger or presumed original state, while the term ‘digital visualisation’ can also be used to predict the future appearance of works of art. Since we cannot guarantee that such digital images will truly correspond to the past or original appearance of artworks, and since ‘digital visualisation’ leaves more openings for varied nuances of interpretation, it is the preferred term that will mostly be used in this dissertation. The focus here is mainly on digital visualisations that look at past appearances of objects. By estimating their original appearance, this approach allows the viewer to visually assess the impact of colour change over time on the aesthetic appearance of the object. \\

The first published digital visualisation was accomplished by Roy S. Berns, in collaboration with the Art Institute of Chicago, on the painting \textit{Un dimanche après-midi à l'île de la Grande Jatte} by Georges Seurat \citep{berns_rejuvenating_2005,berns_rejuvenating_2006}\sidenote{A description of the project is also available at: \url{https://www.cis.rit.edu/people/faculty/berns/seurat/} (accessed 22/09/2020).}. Soon after the rejuvenation of Seurat's painting, a first reconstruction of a Van Gogh painting was made by Roy S. Berns and Ella Hendriks, reversing the fading of a pink ground (eosin and lead white) in \textit{Daubigny’s Garden} (Van Gogh Museum, F765)\sidenote{For the project on the Garden of Daubigny, see also: \url{https://www.cis.rit.edu/people/faculty/berns/research-vangogh-main.html} (accessed 22/09/2020).} \citep[179–80, Figures 7.7, 7.8a and b]{stoner_conservation_2012}. Based on the experience of this visualisation, in 2010 the Van Gogh Museum attempted a second one for the painting \textit{The Bedroom} (Van Gogh Museum, F0482) in the framework of the ‘The Bedroom’ project with later additions in 2013\sidenote{In 2013, a more accurate cochineal pigment was prepared during a reconstruction workshop leading to some modifications of the visualisation made in 2010.}. Later on, a reconstruction of the version kept at the Art Institute of Chicago (F0484) was made and presented during the exhibition \textit{Van Gogh’s Bedrooms} held at the Art Institute of Chicago in 2016 for which a book compiling the results of the whole project has been published with a section dedicated to the consequences of colour changes \citep[89–91]{fiedler_materials_2016}\sidenote{Two videos about the digital visualisation have been made, one by the \citet{rochester_institute_of_technology_artistic_2013} and the other by the \citet{van_gogh_museum_research_2012}.}.\\

From 2014 to 2018, the previously mentioned \gls{REVIGO} project focused on the digital reconstruction of the painting \textit{Field with Irises near Arles} (Van Gogh Museum, F0409). As the painting was undergoing full examination and treatment in the temporary Van Gogh Museum conservation studio in the \gls{RCE} building, this provided the ideal circumstances for collaborative investigation. Surface examination of the painting produced multiple evidence for colour change having taken place, besides the colour descriptions provided in Van Gogh’s letters and in a letter sketch, and historical reproductions of the painting. Reflectance measurements performed during the physical removal of the varnish were used to implement and validate the development of a digital varnish removal software \citep{kirchner_digitally_2018}. Based on multispectral reflectance data, the absorption $K$ and backscattering $S$ parameters of the Kubelka-Munk 2-constant theory were estimated for the whole palette of the painting, \ie 13 pigments, which was then used to calculate the concentration maps for each pigment \citep{kirchner_digitally_2017-1}. The original colours were determined with the use of model paint-outs \citep{kirchner_digitally_2017} for which the authors carefully made pigments and paints combining visual observations, chemical analyses data and historical 19\textsuperscript{th} century paint recipes \citep{geldof_reconstructing_2018}. By acquiring the spectral properties of Van Gogh's paints and developing a transparent methodology, this project prepared the groundwork for future reconstructions.\\

Two further digital reconstructions deserve mention; those performed at the Metropolitan Museum in 2015\sidenote{For an article describing the colour change assessment see \citet{centeno_van_2017}.} and at the Cleveland Museum of Art in 2016 \citep{fieberg_paintings_2017}. Altogether these projects are symptomatic of the growing interest in the technique of digital reconstructions, a trend that can be anticipated to continue into the next decades.\\

\vspace{1cm}

\textit{\underline{Paper-based objects}}


As part of the \gls{REVIGO} project, a group of researchers from TU Delft worked on digital reconstruction of faded drawings \citep{zeng_multi-scale_2019}. Their approach, based on the development of \gls{AI}, required the use of past reproductions of the drawings, for which interpretation of their visual aspect can be doubtful as mentioned earlier. During the project, tension arose between this approach and the material-based methodology followed in the digital reconstruction of the painting \textit{Field with Irises near Arles} (Inv. Nr. F0409, Van Gogh Museum) which negatively impacted the outcomes of the drawing digital reconstruction project. One issue with the \gls{AI} approach is that the computer needs to be trained with ‘valid’ data, a notion that can be quite difficult to define in the case of  cultural heritage, an area where there are many unknowns and subjectivities to be dealt with. Rather than thinking in absolute terms – generally the best we can do is to offer an interpretation. The essence of art and cultural heritage objects rests on the intimate connection held by the observer with a certain civilisation or artist, articulated through the materiality of the object. As a consequence, any scientific research project that ignores this link is prone to failure; in other words, computational tools cannot be developed solely in front of a screen without any observation and contact with the object. On the other hand, the museum community should recognise the rich potential of technology and science and embrace them both with an open mind, making the effort to understand scientific perspective and methodology. Despite the many differences between the two approaches encountered in \gls{REVIGO}, I believe that combining \gls{AI} technology with art technological studies is a relevant idea as a tool to study cultural heritage objects. \\


\newpage
\subsection{Challenges of retracing illumination history}


Just like humans, any object, such as a work of art, has a life: it starts with a time of creation, then follows its own path of evolution, and eventually dies, ending its (useful) life, either destroyed or placed inside a box to be kept. In other words, an analogy can be drawn between the existence of individuals and objects, a concept known as a biographical approach that was first developed in the fields of anthropology and study of material culture \citep{appadurai_social_1986, kopytoff_cultural_1986}. This approach has been applied to cultural heritage objects including archeological artefacts \citep{joy_reinvigorating_2009, pye_engaging_2019} and contemporary artworks \citep{quabeck_reframing_2021, van_de_vall_reflections_2011}. A development of the concept that takes into consideration the ‘mobility of things’ has been proposed by \citet{joyce_things_2015}, known as the ‘object-itinerary’ approach. The latter has, for example, been used by \citet{ford_revisiting_2022} in his PhD research at the National Museum of Art in Oslo, to track movements of the Edvard Munch collection displayed in the Munch Room. \\

Most attempts to recover the history of objects are rooted in a desire to improve our understanding of their current condition. For example, historical studies of past conservation treatments on objects provide valuable information that contributes to current decisions relating to conservation strategies. In the context of Van Gogh, \citet[28–36]{hendriks_vincent_2011} retraced a chronological history of conservation treatments performed on the collection of the Van Gogh Museum, with a focus on the conservation campaign performed by Jan Cornelis Traas between 1926 and 1933 \citep{hendriks_art_2022}. Similarly, the history of exhibitions for an object or a group of objects can be traced. Such a study would provide valuable historical information on several levels, including the history of the work, museum studies, and the diffusion of artistic ideas. From a conservation point of view, this information could also allow us to explain certain forms of degradation or traces of damage. For example, a work of art that has travelled regularly and been exposed frequently to fluctuating climatic conditions has necessarily been subjected to a high level of stress, with visible consequences for the appearance of the work. Certain details, made visible under appropriate conditions, may indicate exposure to a high level of mechanical stress, or provide information to unravel the process of fabricating the object\sidenote{This analogy between the work of a detective and that of an art historian or conservator has been studied extensively by \citet{ginzburg_morelli_1980}.}. \\

\newpage

The exhibition history of a given work only partially corresponds to its illumination history however. In other words, the display of works in temporary and permanent exhibitions only represents a small fraction of the total light exposure dose history. Assuming that we can identify all the exhibitions in which a work has participated, quantifying the dose of light energy received by the work at all these venues requires several assumptions (light intensity, duration of exposure, light spectrum) to be made, usually with a high degree of uncertainty\sidenote{However, recent lighting installations enable precise control and monitoring of lighting conditions, so that ultimately museums can precisely calculate the annual total exposure dose received by each object on display. This is the case for the Van Gogh Museum in Amsterdam: Kees van den Meiracker, personal communication, 21/01/2020.}. Furthermore, the assumption that the colour change of a work is mainly caused by exposure to optical radiation is not always valid. There are other factors than just light  leading to colour change and it is often difficult to distinguish the share of each one in the total colour change. This means that exposure of objects on display in galleries and museums only accounts for a small, if not negligible part of the total light-induced colour change. In the case of Van Gogh, this assertion is backed up by information in his correspondence. For instance, we know that the yellow house he rented in Arles was very bright and well exposed to the sun\sidenote{Letters n$^\circ$602 and n$^\circ$612 (\url{https://vangoghletters.org/vg/;} accessed 12/03/2023).}, which tells us that the paintings made and hung on the light-reflective, whitewashed walls of the house with studio, probably received a large dose of light energy early on. Other aspects of his working practice played a role, as he mentions that he sometimes laid his paintings outside to dry in the sun when he lived in the South of France\sidenote{Letters n$^\circ$696 and n$^\circ$780 (\url{https://vangoghletters.org/vg/;} accessed 12/03/2023).}.\\

If a historical approach is limited to determining the extent to which Van Gogh’s work was submitted to light, cross-referencing the results of historical information about the exhibitions of works of art with scientific analyses may be relevant. In his PhD dissertation, \citet[318–25]{del_hoyo-melendez_study_2010} developed a scientific methodology to estimate the total dose of light energy received by an object. By performing microfading analyses on degraded as well as light-shielded areas, he was able to quantify the exposure dose received by the paper of an envelope. It would be most interesting to apply his methodology to a range of materials and objects, but unfortunately time constraints meant that this could not be pursued within the context of this PhD.\\


\newpage
\section{Conclusion}

This introductory chapter has laid out a conceptual framework and historical foundation for the challenge of light management in cultural institutions and light-induced colour change studies on Van Gogh’s works respectively. \\ 

Over the years, theoretical and practical tools to manage the risk of colour change have been developed and integrated into conservation practices. The challenge of light-induced colour change on objects is treated from a risk management perspective for which standard procedures have been established. Altogether, it creates a coherent framework within which museum professionals can handle colour change issues with a certain degree of flexibility so that appropriate measures can be taken according to the context in place. \\

Numerous studies and projects have focused on colour changes in Van Gogh's works, from which three main outcomes can be stated. Firstly, a good estimation of the palette used for Van Gogh’s paintings and works on paper has been acquired. Secondly, working methodologies, especially involving digital visualisation and light-ageing experiments have been established, providing a solid groundwork upon which future research projects can build. Finally, the most light-sensitive pigments have been investigated, and their degradation paths and colour change behaviour have been characterised. Altogether, these research projects have promoted awareness and understanding of colour change among cultural heritage professionals, prompting museums to implement or adapt their lighting policy in view of the risk of future colour change. However, in the context of Van Gogh’s works, there is still a lack of colour change data obtained on original paintings and drawings that can support digital visualisations of predicted light-induced colour change. This doctoral thesis continues the line of previous research, building on the experiences and results gained since the first research on this subject took place. It focuses on the application and development of microfadeometry for the study of colour change in Van Gogh's work, continuing the initial attempts made by \citet{fieberg_paintings_2017}. \\
